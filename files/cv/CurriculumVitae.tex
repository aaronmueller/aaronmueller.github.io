%%%%%%%%%%%%%%%%%%%%%%%%%%%%%%%%%%%%%%%%%%%%%%%%%%%%%%%%%%%%%%%%%%%%%%%%
%%%%%%%%%%%%%%%%%%%%%% Simple LaTeX CV Template %%%%%%%%%%%%%%%%%%%%%%%%
%%%%%%%%%%%%%%%%%%%%%%%%%%%%%%%%%%%%%%%%%%%%%%%%%%%%%%%%%%%%%%%%%%%%%%%%

%%%%%%%%%%%%%%%%%%%%%%%%%%%%%%%%%%%%%%%%%%%%%%%%%%%%%%%%%%%%%%%%%%%%%%%%
%% NOTE: If you find that it says                                     %%
%%                                                                    %%
%%                           1 of ??                                  %%
%%                                                                    %%
%% at the bottom of your first page, this means that the AUX file     %%
%% was not available when you ran LaTeX on this source. Simply RERUN  %%
%% LaTeX to get the ``??'' replaced with the number of the last page  %%
%% of the document. The AUX file will be generated on the first run   %%
%% of LaTeX and used on the second run to fill in all of the          %%
%% references.                                                        %%
%%%%%%%%%%%%%%%%%%%%%%%%%%%%%%%%%%%%%%%%%%%%%%%%%%%%%%%%%%%%%%%%%%%%%%%%

%%%%%%%%%%%%%%%%%%%%%%%%%%%% Document Setup %%%%%%%%%%%%%%%%%%%%%%%%%%%%

% Don't like 10pt? Try 11pt or 12pt
\documentclass[10pt]{article}
\RequirePackage[T1]{fontenc}
\usepackage[utf8]{inputenc}

% LaTeX will typeset using Computer Modern Roman, which a lot of
% non-mathematicians and non-engineers won't like. Also, a few PDF
% viewers may not render CMR very well. Instead, Times New Roman can
% be used. That's what this package does.
\usepackage{times}

% The automated optical recognition software used to digitize resume
% information works best with fonts that do not have serifs. This
% command uses a sans serif font throughout. Uncomment both lines (or at
% least the second) to restore a Roman font (i.e., a font with serifs).
% (NOTE: This requires the times package above)
%\renewcommand{\familydefault}{\sfdefault}

% This is a helpful package that puts math inside length specifications
\usepackage{calc}

% This package helps LaTeX auto-hyphenate hyphenated words if you use
% special hyphens. For example, bio\-/mimicry will properly hyphenate
% ``mimicry'' if necessary.
\usepackage[shortcuts]{extdash}

% Layout: Puts the section titles on left side of page
\reversemarginpar

%
%         PAPER SIZE, PAGE NUMBER, AND DOCUMENT LAYOUT NOTES:
%
% The next \usepackage line changes the layout for CV style section
% headings as marginal notes. It also sets up the paper size as either
% letter or A4. By default, letter was used. If A4 paper is desired,
% comment out the letterpaper lines and uncomment the a4paper lines.
%
% As you can see, the margin widths and section title widths can be
% easily adjusted.
%
% ALSO: Notice that the includefoot option can be commented OUT in order
% to put the PAGE NUMBER *IN* the bottom margin. This will make the
% effective text area larger.
%
% IF YOU WISH TO REMOVE THE ``of LASTPAGE'' next to each page number,
% see the note about the +LP and -LP lines below. Comment out the +LP
% and uncomment the -LP.
%
% IF YOU WISH TO REMOVE PAGE NUMBERS, be sure that the includefoot line
% is uncommented and ALSO uncomment the \pagestyle{empty} a few lines
% below.
%

%% Use these lines for letter-sized paper
\usepackage[paper=letterpaper,
%includefoot, % Uncomment to put page number above margin
marginparwidth=1.0in,     % Length of section titles
marginparsep=.05in,       % Space between titles and text
margin=0.9in,               % 1 inch margins
includemp]{geometry}

%% Use these lines for A4-sized paper
%\usepackage[paper=a4paper,
%            %includefoot, % Uncomment to put page number above margin
%            marginparwidth=30.5mm,    % Length of section titles
%            marginparsep=1.5mm,       % Space between titles and text
%            margin=25mm,              % 25mm margins
%            includemp]{geometry}

%% More layout: Get rid of indenting throughout entire document
\setlength{\parindent}{0in}

% Provides special list environments and macros to create new ones
\usepackage[shortlabels]{enumitem}

% Simpler bibsections for CV sections
% (thanks to natbib for inspiration)
%
% * For lists of references with hanging indents and no numbers:
%
%   \begin{bibsection}
%       \item ...
%   \end{bibsection}
%
% * For numbered lists of references (with hanging indents):
%
%   \begin{bibenum}
%       \item ...
%   \end{bibenum}
%
%   Note that bibenum numbers continuously throughout. To reset the
%   counter, use
%
%   \restartlist{bibenum}
%
%   at the place where you want the numbering to reset.

\makeatletter
\newlength{\bibhang}
\setlength{\bibhang}{1em}
\newlength{\bibsep}
{\@listi \global\bibsep\itemsep \global\advance\bibsep by\parsep}
\newlist{bibsection}{itemize}{3}
\setlist[bibsection]{label=,leftmargin=\bibhang,%
	itemindent=-\bibhang,
	itemsep=\bibsep,parsep=\z@,partopsep=0pt,
	topsep=0pt}
\newlist{bibenum}{enumerate}{3}
\setlist[bibenum]{label=[\arabic*],resume,leftmargin={\bibhang+\widthof{[999]}},%
	itemindent=-\bibhang,
	itemsep=\bibsep,parsep=\z@,partopsep=0pt,
	topsep=0pt}
\let\oldendbibenum\endbibenum
\def\endbibenum{\oldendbibenum\vspace{-.6\baselineskip}}
\let\oldendbibsection\endbibsection
\def\endbibsection{\oldendbibsection\vspace{-.6\baselineskip}}
\makeatother

%% Reference the last page in the page number
%
% NOTE: comment the +LP line and uncomment the -LP line to have page
%       numbers without the ``of ##'' last page reference)
%
% NOTE: uncomment the \pagestyle{empty} line to get rid of all page
%       numbers (make sure includefoot is commented out above)
%
\usepackage{fancyhdr,lastpage}
\pagestyle{fancy}
%\pagestyle{empty}      % Uncomment this to get rid of page numbers
\fancyhf{}\renewcommand{\headrulewidth}{0pt}
\fancyfootoffset{\marginparsep+\marginparwidth}
\newlength{\footpageshift}
\setlength{\footpageshift}
{0.5\textwidth+0.5\marginparsep+0.5\marginparwidth-2in}
\lfoot{\hspace{\footpageshift}%
	\parbox{4in}{\, \hfill %
		\arabic{page} of \protect\pageref*{LastPage} % +LP
		%                    \arabic{page}                               % -LP
		\hfill \,}}

% Finally, give us PDF bookmarks
\usepackage{color,hyperref}
\definecolor{darkblue}{rgb}{0.0,0.0,0.3}
\hypersetup{colorlinks,breaklinks,
	linkcolor=darkblue,urlcolor=darkblue,
	anchorcolor=darkblue,citecolor=darkblue}

%%%%%%%%%%%%%%%%%%%%%%%% End Document Setup %%%%%%%%%%%%%%%%%%%%%%%%%%%%


%%%%%%%%%%%%%%%%%%%%%%%%%%% Helper Commands %%%%%%%%%%%%%%%%%%%%%%%%%%%%

%%% HEADING AT TOP OF CURRICULUM VITAE

% The title (name) with a horizontal rule under it
% (optional argument typesets an object right-justified across from name
%  as well)
%
% Usage: \makeheading{name}
%        OR
%        \makeheading[right_object]{name}
%
% Place at top of document. It should be the first thing.
% If ``right_object'' is provided in the square-braced optional
% argument, it will be right justified on the same line as ``name'' at
% the top of the CV. For example:
%
%       \makeheading[\emph{Curriculum vitae}]{Your Name}
%
% will put an emphasized ``Curriculum vitae'' at the top of the document
% as a title. Likewise, a picture could be included:
%
%   \makeheading[{\includegraphics[height=1.5in]{my_picture}}]{Your Name}
%
% the picture will be flush right across from the name. For this example
% to work, make sure the extra set of curly braces is included. Also
% makes ure that \usepackage{graphicx} is somewhere in the preamble.
\newcommand{\makeheading}[2][]%
{\hspace*{-\marginparsep minus \marginparwidth}%
	\begin{minipage}[t]{\textwidth+\marginparwidth+\marginparsep}%
		{\large \bfseries #2 \hfill #1}\\[-0.15\baselineskip]%
		\rule{\columnwidth}{1pt}%
\end{minipage}}

%%% SECTION HEADINGS

% The section headings. Flush left in small caps down pseudo-margin.
%
% Usage: \section{section name}
\renewcommand{\section}[1]{\pagebreak[3]%
	\vspace{0.5\baselineskip}%
	\phantomsection\addcontentsline{toc}{section}{#1}%
	\noindent\llap{\scshape\smash{\parbox[t]{\marginparwidth}{\hyphenpenalty=10000\raggedright #1}}}%
	\vspace{-\baselineskip}\par}

%%% LISTS

% This macro alters a list by removing some of the space that follows the list
% (is used by lists below)
\newcommand*\fixendlist[1]{%
	\expandafter\let\csname preFixEndListend#1\expandafter\endcsname\csname end#1\endcsname
	\expandafter\def\csname end#1\endcsname{\csname preFixEndListend#1\endcsname\vspace{-0.6\baselineskip}}}

% These macros help ensure that items in outer-type lists do not get
% separated from the next line by a page break
% (they are used by lists below)
\let\originalItem\item
\newcommand*\fixouterlist[1]{%
	\expandafter\let\csname preFixOuterList#1\expandafter\endcsname\csname #1\endcsname
	\expandafter\def\csname #1\endcsname{\let\oldItem\item\def\item{\pagebreak[2]\oldItem}\csname preFixOuterList#1\endcsname}
	\expandafter\let\csname preFixOuterListend#1\expandafter\endcsname\csname end#1\endcsname
	\expandafter\def\csname end#1\endcsname{\let\item\oldItem\csname preFixOuterListend#1\endcsname}}
\newcommand*\fixinnerlist[1]{%
	\expandafter\let\csname preFixInnerList#1\expandafter\endcsname\csname #1\endcsname
	\expandafter\def\csname #1\endcsname{\let\oldItem\item\let\item\originalItem\csname preFixInnerList#1\endcsname}
	\expandafter\let\csname preFixInnerListend#1\expandafter\endcsname\csname end#1\endcsname
	\expandafter\def\csname end#1\endcsname{\csname preFixInnerListend#1\endcsname\let\item\oldItem}}

% An itemize-style list with lots of space between items
%
% Usage:
%   \begin{outerlist}
%       \item ...    % (or \item[] for no bullet)
%   \end{outerlist}
\newlist{outerlist}{itemize}{3}
\setlist[outerlist]{label=\enskip\textbullet,leftmargin=*}
\fixendlist{outerlist}
\fixouterlist{outerlist}

% An environment IDENTICAL to outerlist that has better pre-list spacing
% when used as the first thing in a \section
%
% Usage:
%   \begin{lonelist}
%       \item ...    % (or \item[] for no bullet)
%   \end{lonelist}
\newlist{lonelist}{itemize}{3}
\setlist[lonelist]{label=\enskip\textbullet,leftmargin=*,partopsep=0pt,topsep=0pt}
\fixendlist{lonelist}
\fixouterlist{lonelist}

% An itemize-style list with little space between items
%
% Usage:
%   \begin{innerlist}
%       \item ...    % (or \item[] for no bullet)
%   \end{innerlist}
\newlist{innerlist}{itemize}{3}
\setlist[innerlist]{label=\enskip--,leftmargin=*,parsep=0pt,itemsep=0pt,topsep=0pt,partopsep=0pt}
\fixinnerlist{innerlist}

% An environment IDENTICAL to innerlist that has better pre-list spacing
% when used as the first thing in a \section
%
% Usage:
%   \begin{loneinnerlist}
%       \item ...    % (or \item[] for no bullet)
%   \end{loneinnerlist}
\newlist{loneinnerlist}{itemize}{3}
\setlist[loneinnerlist]{label=\enskip\textbullet,leftmargin=*,parsep=0pt,itemsep=0pt,topsep=0pt,partopsep=0pt}
\fixendlist{loneinnerlist}
\fixinnerlist{loneinnerlist}

%%% EXTRA SPACE

% To add some paragraph space between lines.
% This also tells LaTeX to preferably break a page on one of these gaps
% if there is a needed pagebreak nearby.
\newcommand{\blankline}{\quad\pagebreak[3]}
\newcommand{\halfblankline}{\quad\vspace{-0.5\baselineskip}\pagebreak[3]}

%%% FORMATTING MACROS

% Provides a linked \doi{#1} that links doi:#1 to http://dx.doi.org/#1
\usepackage{doi}
% To change the text before the DOI, adjust this command
%\renewcommand\doitext{doi:}

% Provides a linked \url{#1} that doesn't require escape characters
\usepackage{url}

% You can adjust the style \url{} uses here:
% (options are: same, rm, sf, tt; defaults to tt)
\urlstyle{same}

% For \email{ADDRESS}, links ADDRESS to the url mailto:ADDRESS
% (uncomment to typeset the e\-/mail address in typewriter font;
%  otherwise, will be typeset in the \urlstyle above)
%\DeclareUrlCommand\emaillink{\urlstyle{tt}}
\providecommand*\emaillink[1]{\nolinkurl{#1}}
\providecommand*\email[1]{\href{mailto:#1}{\emaillink{#1}}}

\providecommand\BibTeX{{B\kern-.05em{\sc i\kern-.025em b}\kern-.08em \TeX}}
\providecommand\Matlab{\textsc{Matlab}}

% Custom hyphenation rules for words that LaTeX has trouble with
\hyphenation{bio-mim-ic-ry bio-in-spi-ra-tion re-us-a-ble pro-vid-er Media-Wiki}

%%%%%%%%%%%%%%%%%%%%%%%% End Helper Commands %%%%%%%%%%%%%%%%%%%%%%%%%%%

%%%%%%%%%%%%%%%%%%%%%%%%% Begin CV Document %%%%%%%%%%%%%%%%%%%%%%%%%%%%

\begin{document}
	\makeheading{\LARGE{Aaron M. Mueller}}
	\\

	\section{Contact}
	Department of Computer Science
	\hfill{\textit{Phone:} +1-502-550-4938} \\
	Johns Hopkins University
	\hfill{\textit{E-mail:} amueller@jhu.edu} \\
	3400 N. Charles St., Hackerman 319
	\hfill{\textit{Website:} aaronmueller.github.io} \\
	Baltimore, MD 21218-2691 (U.S.A.)
	
	%%
	%% In modern CV's, it seems like ``Objective'' is frowned upon. Instead,
	%% incorporate it into a well-constructed cover letter. The ``More
	%% information'' can go at the end of the CV, but it should not distract
	%% from the section giving references available to contact.
	%%
	%
	% \section{Objective}
	%
	% Placement in an academic position (i.e., faculty, postdoctoral, or
	% research scientist) that allows for advanced research in distributed
	% complex adaptive systems (i.e., modeling, analysis, design, and
	% verification) with a particular focus on the control of engineered
	% agents (e.g., for communications, control, software, electronics, and
	% sustainability) and the analysis of biological phenomena (e.g.,
	% self-organization, ecological rationality)
	% \begin{innerlist}
	% \item More information and auxiliary documents can be found at\\\url{http://www.tedpavlic.com/facjobsearch/}
	% \end{innerlist}
	
	\section{Research Interests}
	\begin{itemize}[topsep=0pt]
		\setlength{\itemindent}{-.2in}
		\itemsep-0.3em
		\item Machine translation
		\item Endangered and low-resource language varieties
		\item Syntax/semantics interface
		\item Pragmatics and sociolinguistics
	\end{itemize}
	\section{Education}

	{\textbf{Johns Hopkins University}},
	Baltimore, MD
	\begin{outerlist}[topsep=0pt]
	\itemsep-0.3em
	\item[] PhD student, Computer Science (Whiting School of Engineering).
	\hfill \textbf{August 2018 - Present}
	\item[] M.S.E., Computer Science (Whiting School of Engineering).
	\hfill Expected \textbf{May 2020}
	\item[] \textit{Affiliation}: Center for Language and Speech Processing.
	\item[] \textit{Advisor}: David Yarowsky.\\
	\end{outerlist}

	{\textbf{University of Kentucky}},
	Lexington, KY
	\begin{outerlist}[topsep=0pt]
		\itemsep-0.3em
		\item[] B.S., Computer Science (College of Engineering). Honors.
		\hfill \textbf{May 2018}
		\item[] B.S., Linguistics (College of Arts \& Sciences), Honors.
		\hfill \textbf{May 2018}
		\item[] GPA: 4.0. \textit{Summa cum laude}.
		\item[] \emph{Thesis}: Neural Machine Translation for Canadian French.
		\item[] \emph{Advisors}: Mark Richard Lauersdorf, Ramakanth Kavuluru.
	\end{outerlist}

	\halfblankline
		
	\section{Research Experience}

	{\textbf{University of Kentucky}},
	Lexington, KY\\
	\textbf{Institutional Research \& Advanced Analytics}\\
		\textit{Research Assistant}
		\hfill \textbf{May 2018 - August 2018}\\
		Supervisors: Nathan Jacobs, Craig Rudick
		\begin{innerlist}
			\item Designed a system that, given a student's prior performance,
			predicts their future grades course-by-course.
			\item Employed deep LSTMs and NLP techniques (e.g., learning word embeddings
			on course descriptions) to learn student and course profiles.
			\item Designed an ordinal loss function that observed
			how close the prediction was to the real grade.
		\end{innerlist}

	\halfblankline

	{\textbf{University of Massachusetts Amherst}},
	Amherst, MA\\
	\textbf{Statistical Social Language Analysis Lab}\\
		\textit{Research Assistant}%
		\hfill \textbf{May 2017 - August 2017}\\
		Supervisor: Brendan O'Connor
		\begin{innerlist}
			\item Collaborated with graduate students and faculty to improve an entity-event extraction system.
			\item Given news articles, used distant supervision to extract the names of American citizens who have been killed by police.
			\item Integrated entity mention and relation systems into pre-existing sentence-level model.
			\item Integrated document-level information (e.g., noun coreference) as features to improve our system.
		\end{innerlist}
	
	\halfblankline
	
	{\textbf{University of Kentucky}},
	Lexington, KY\\
	\textbf{Linguistics Department}\\
		\textit{Juried Project}%
		\hfill \textbf{August 2016 - May 2017}\\
		Jurors: Hilaria Cruz, Raphael Finkel, Phil Harling
		\begin{innerlist}
			\item Preservation and revitalization of a low-resource indigenous language---Chatino---spoken natively by approximately 40,000 individuals.
			\item Automatic speech recognition system training with Sphinx.
			\item Promotion and creation of Chatino language-learning resources, including open-source speech data, a speech corpus, and a website.
		\end{innerlist}
	
	\halfblankline
	
		\textit{Directed Independent Research}
		\hfill \textbf{September 2015 - May 2016}\\
		Supervisor: Mark Richard Lauersdorf
		\begin{innerlist}
			\item Exploration of the manner in which societal perceptions and opinions of gender and LGBT communities are expressed linguistically.
			\item Application of corpus-based statistical methods to the analysis of diachronic sociolinguistic data.
		\end{innerlist}

	\halfblankline
	
	{\textbf{Johns Hopkins University}},
	Baltimore, MD\\
	\textbf{Center for Language \& Speech Processing}\\
		\textit{Research Assistant}%
		\hfill \textbf{May 2016 - August 2016}\\
		Supervisor: David Yarowsky
		\begin{innerlist}
			\item Collaborated with undergraduate and graduate students, post-doctoral researchers, and faculty to create automatic translation systems for low-resource languages.
			\item Assisted in the creation of a lemma-based English-Uyghur translation model.
			\item Built a morphological generator for Crimean Tatar.
			\item Created parsers to extract translation tables from foreign editions of Wiktionary. 
			\item Funded by DARPA's Low-Resource Languages for Emergent Incidents (LORELEI) program.
		\end{innerlist}

	\section{Fellowships and Awards}

	\textit{National Science Foundation Graduate Fellowship}
	\hfill{\bf{2018 - 2023}}
	
	\halfblankline
	
	\textit{Gaines Fellow}
	\hfill{\bf{2016 - 2018}}\\
		Two-year fellowship awarded tovundergraduates with outstanding academic performance, demonstrated ability to conduct undergraduate research, an interest in public issues, and a desire to enhance understanding of the human condition through the humanities. Requires the completion of a juried project, a major thesis project, and a seminar in the humanities. (\$5,000)
	
	\halfblankline

	\textit{Patterson Scholar}
	\hfill{\bf{2014 - 2018}}\\
		Granted to University of Kentucky students who have earned National Merit Finalist standing and who maintain a 3.300 cumulative GPA. (\$80,000+)

	\halfblankline

	\textit{Raymond F. Betts Scholar}
	\hfill{\bf{2017}}\\
		Awarded to rising seniors conducting thesis research, especially if travel is
		necessary. Used funds to study Québec French and in Montréal and Québec City during
		winter of 2017--2018. (\$2,500)

	\halfblankline

	\textit{Goldwater Scholarship (Honorable Mention)}
	\hfill{\bf{2017}}
	
	\halfblankline
	
	\textit{Phi Beta Kappa}
	\hfill{\bf{2017}}
	
	\halfblankline
	
	\textit{National Merit Scholar}
	\hfill{\bf{2014}}
	
	\section{Selected Presentations}

	Mueller, Aaron; Keith, Katherine; Handler, Abe; Blodgett, Su Lin; and O'Connor, Brendan. The Identification of Civilians Killed by Police with Supervised Entity-Event Extraction. \emph{2017 UMass Amherst Research Experience for Undergraduates (REU) Showcase}, Amherst, MA, August 9, 2017.

	\halfblankline

	Mueller, Aaron; Finkel, Raphael; and Cruz, Hilaria. Documenting and Promoting the Chatino Language and Orthography. \emph{Juried Presentation in Satisfaction of the Requirements of the Gaines Fellowship}, Lexington, KY, February 21, 2017.
	
	\halfblankline

	Mueller, Aaron. A Lemma-Based Approach for English-Uyghur Statistical Machine Translation. \emph{9th Annual Conference of the Illinois Language and Linguistics Society (ILLS9)}, Urbana, IL, March 31--April 1, 2017.
	
	\halfblankline

	Mueller, Aaron. Lexical and Semantic Shifts in the Linguistic Construction of Social Gender: A Corpus-Based Analysis of Written U.S. English. Poster.
	\emph{9th Annual Toronto Undergraduate Linguistics Conference (TULCON9)}, Toronto, ON, March 4--6, 2016.

	\section{Additional Experience}

	{\textbf{Chellgren Center for Undergraduate Excellence}},
	Lexington, KY\\
		\textit{Student Outreach Team}
		\hfill \textbf{August 2017 - May 2018}
		\begin{innerlist}
			\item Promoted the Office of Undergraduate Research and the Chellgren Fellowship.
			\item Organized panel discussions, workshops, and lectures on
			abstract writing, research poster design, and academic presentations.
			\item Led the review committee for our Showcase of Undergraduate Scholars,
			an annual conference where University of Kentucky undergraduates present their
			research to other students and faculty.
		\end{innerlist}

	\halfblankline

	{\textbf{UK Special Collections Research Center}},
	Lexington, KY\\
		\textit{Archiving Intern}%
		\hfill \textbf{September 2015 - May 2016}
		\begin{innerlist}
			\item Created a diachronic corpus of Lexingtonian architectural documentation which includes zoning ordinances and Lexington Planning Committee meeting minutes.
			\item Made accessible an archival collection that highlights Lexington's architectural history.
			\item Cooperated on a multiformat project that layered GIS technology, city government data, archival photos, and digital humanities tools.
		\end{innerlist}
	
	\section{Volunteer Service}
	
	\textbf{Amnesty International}
	\hfill{\textbf{2012 - Present}}
	\begin{innerlist}
		\item International organization dedicated to the protection of human rights worldwide.
		\item Member of Amnesty International at Johns Hopkins and Washington, D.C.
		\item \textit{Founder} and former \textit{Co-President} of Amnesty International at the University of Kentucky.
		\item Organize fundraisers and informational events: concerts featuring local bands, panel discussions on contemporary issues, faculty debates, petitions, protests, and more.
		\item Attendee at Amnesty's national Human Rights Conference.
	\end{innerlist}
	
	\halfblankline
	
	\textbf{Oxfam International}
	\hfill{\textbf{2014 - Present}}
	\begin{innerlist}
		\item International organization dedicated to poverty eradication, disaster relief, and political advocacy for those facing harsh living conditions and injustice
		worldwide.
		\item Member of Oxfam in Washington, D.C.
		\item \textit{Founder} and former \textit{Co-Coordinator} of Oxfam at the University of Kentucky.
		\item Plan awareness-raising, fundraising, and petitioning events for Oxfam's GROW campaign with a distinct focus on food inequality in impoverished and war-stricken countries.
	\end{innerlist}

	\halfblankline
	
	\textbf{Society for the Promotion of Undergraduate Research}
	\hfill{\textbf{2014 - 2017}}
	\begin{innerlist}
		\item University-of-Kentucky-based organization founded to assist undergraduates in finding research opportunities and in bettering their research.
		\item \textit{Co-President} and former \textit{Director of Events}.
		\item Plan and execute the Showcase of Undergraduate Scholars.
		\item Plan research workshops to help undergraduates improve their abstracts, presentations, and papers.
		\item Organize faculty panel discussions, promotional events, and information sessions on getting involved in undergraduate research and on scholarship and summer research opportunities.
		\item Merged with the Student Outreach Team in May 2017 (joined upon merge).
	\end{innerlist}

	\section{Computing Skills}
	
	Computer Programming:
	%
	\begin{innerlist}
		\item Languages: Python, C, C$+$$+$, Java, HTML, CSS, Javascript, \Matlab, R, UNIX shell scripting, GNU make.
		\item Development Environments: Vim, Emacs, Eclipse.
		\item Version Control: DVCS (Git, Bitbucket).
		\item Machine Learning Toolkits: TensorFlow, Sockeye, Keras, PyTorch, OpenNMT.
		\item Operating Systems: Linux (Debian, Ubuntu, CentOS, Arch), Microsoft Windows family, macOS, Android, iOS.
	\end{innerlist}
	
	\halfblankline
	
	Linguistic Tools:
	\begin{innerlist}
		\item[] Praat, AntConc, QGIS, Audacity.
	\end{innerlist}

	\halfblankline
	
	Desktop Editing and Productivity Software:
	%
	\begin{innerlist}
		\item \TeX{} (\LaTeX{}, \BibTeX{}).
		\item GIMP, Inkscape.
		\item Microsoft Office, OpenOffice.org, LibreOffice, G Suite.
	\end{innerlist}

	\section{Languages}
	Fluent in English and French (Québec). Rudimentary Inuktitut and Dutch. Experience
	with San Juan Quiahije Chatino, Uyghur, and Crimean Tatar through research.

	\section{Interests and Hobbies}

	\begin{itemize}[topsep=0pt]
		\setlength{\itemindent}{-.2in}
		\itemsep-0.3em
		\item[--] Music: guitar (acoustic, classical, electric), bass, banjo,
		synthesizer, MaxMSP, Ableton Live.
		\item[--] Athletics: contra, swing dance, cold-weather hiking.
		\item[--] Hobbies: natural language learning, reading (philosophy, politics), city exploration.
		\item[--] Cooking: Thai, French, German, Southern.
	\end{itemize}

	%\section{References Available to Contact}
	%
	%\textbf{Dr. Mark Richard Lauersdorf}
	%(e\-/mail:~lauersdorf@uky.edu; phone:~+1-859-257-7101)
	%
	%\begin{innerlist}
	%	\item Associate Professor of Linguistics,
	%	University of Kentucky
		
		%\item[$\diamond$] Linguistics Program,
		%1215 Patterson Office Tower,
		%University of Kentucky,
		%Lexington, KY 40506-0027
		
	%	\item[$\star$] \emph{Dr.~Lauersdorf is a current thesis advisor, former sociolinguistics research advisor, and former professor of multiple classes of mine.}
	%\end{innerlist}
	
	%\halfblankline
	
	%\textbf{Dr. Jerzy Jaromczyk}
	%(e\-/mail:~jurek@cs.uky.edu; phone:~+1-859-257-1186)
	%
	%\begin{innerlist}
	%	\item Associate Professor of Computer Science,
	%	University of Kentucky
		
		%\item[$\diamond$] Davis Marksbury Building,
		%329 Rose Street,
		%Lexington, KY 40506-0633
		
	%	\item[$\star$] \emph{Dr.~Jaromczyk is a current computer science research advisor.}
	%\end{innerlist}
	
	%\halfblankline
	%
	%\textbf{Dr. Hilaria Cruz}
	%(e\-/mail:~hcr224@g.uky.edu)
	%
	%\begin{innerlist}
	%	\item Lyman T. Johnson Postdoctoral Fellow,
	%	University of Kentucky
		
		%\item[$\diamond$] Linguistics Program,
		%1215 Patterson Office Tower,
		%University of Kentucky,
		%Lexington, KY 40506-0027
		
	%	\item[$\star$] \emph{Dr.~Cruz juried a language revitalization project of mine. She is a native speaker of Chatino, a low-resource indigenous language.}
	%\end{innerlist}
	
	%\halfblankline
	
	% The ``More Info'' section may not be necessary; make sure it's short
	% so it doesn't prevent people from seeing references available to
	% contact.
	%\section{More Information}
	
	%More information can be found at\\	\url{https://www.linkedin.com/in/aaron-mueller-8b9334104}.
	
\end{document}

%%%%%%%%%%%%%%%%%%%%%%%%%% End CV Document %%%%%%%%%%%%%%%%%%%%%%%%%%%%%

%----------------------------------------------------------------------%
% The following is copyright and licensing information for
% redistribution of this LaTeX source code; it also includes a liability
% statement. If this source code is not being redistributed to others,
% it may be omitted. It has no effect on the function of the above code.
%----------------------------------------------------------------------%
% Copyright (c) 2007, 2008, 2009, 2010, 2011 by Theodore P. Pavlic
%
% Unless otherwise expressly stated, this work is licensed under the
% Creative Commons Attribution-Noncommercial 3.0 United States License. To
% view a copy of this license, visit
% http://creativecommons.org/licenses/by-nc/3.0/us/ or send a letter to
% Creative Commons, 171 Second Street, Suite 300, San Francisco,
% California, 94105, USA.
%
% THE SOFTWARE IS PROVIDED "AS IS", WITHOUT WARRANTY OF ANY KIND, EXPRESS
% OR IMPLIED, INCLUDING BUT NOT LIMITED TO THE WARRANTIES OF
% MERCHANTABILITY, FITNESS FOR A PARTICULAR PURPOSE AND NONINFRINGEMENT.
% IN NO EVENT SHALL THE AUTHORS OR COPYRIGHT HOLDERS BE LIABLE FOR ANY
% CLAIM, DAMAGES OR OTHER LIABILITY, WHETHER IN AN ACTION OF CONTRACT,
% TORT OR OTHERWISE, ARISING FROM, OUT OF OR IN CONNECTION WITH THE
% SOFTWARE OR THE USE OR OTHER DEALINGS IN THE SOFTWARE.