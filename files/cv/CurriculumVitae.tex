%%%%%%%%%%%%%%%%%%%%%%%%%%%%%%%%%%%%%%%%%%%%%%%%%%%%%%%%%%%%%%%%%%%%%%%%
%%%%%%%%%%%%%%%%%%%%%% Simple LaTeX CV Template %%%%%%%%%%%%%%%%%%%%%%%%
%%%%%%%%%%%%%%%%%%%%%%%%%%%%%%%%%%%%%%%%%%%%%%%%%%%%%%%%%%%%%%%%%%%%%%%%

%%%%%%%%%%%%%%%%%%%%%%%%%%%%%%%%%%%%%%%%%%%%%%%%%%%%%%%%%%%%%%%%%%%%%%%%
%% NOTE: If you find that it says                                     %%
%%                                                                    %%
%%                           1 of ??                                  %%
%%                                                                    %%
%% at the bottom of your first page, this means that the AUX file     %%
%% was not available when you ran LaTeX on this source. Simply RERUN  %%
%% LaTeX to get the ``??'' replaced with the number of the last page  %%
%% of the document. The AUX file will be generated on the first run   %%
%% of LaTeX and used on the second run to fill in all of the          %%
%% references.                                                        %%
%%%%%%%%%%%%%%%%%%%%%%%%%%%%%%%%%%%%%%%%%%%%%%%%%%%%%%%%%%%%%%%%%%%%%%%%

%%%%%%%%%%%%%%%%%%%%%%%%%%%% Document Setup %%%%%%%%%%%%%%%%%%%%%%%%%%%%

% Don't like 10pt? Try 11pt or 12pt
\documentclass[10pt]{article}
\RequirePackage[T1]{fontenc}
\usepackage[utf8]{inputenc}

% LaTeX will typeset using Computer Modern Roman, which a lot of
% non-mathematicians and non-engineers won't like. Also, a few PDF
% viewers may not render CMR very well. Instead, Times New Roman can
% be used. That's what this package does.
\usepackage{times}
\usepackage{inconsolata}

% The automated optical recognition software used to digitize resume
% information works best with fonts that do not have serifs. This
% command uses a sans serif font throughout. Uncomment both lines (or at
% least the second) to restore a Roman font (i.e., a font with serifs).
% (NOTE: This requires the times package above)
%\renewcommand{\familydefault}{\sfdefault}

% This is a helpful package that puts math inside length specifications
\usepackage{calc}

% This package helps LaTeX auto-hyphenate hyphenated words if you use
% special hyphens. For example, bio\-/mimicry will properly hyphenate
% ``mimicry'' if necessary.
\usepackage[shortcuts]{extdash}

% Layout: Puts the section titles on left side of page
\reversemarginpar

%
%         PAPER SIZE, PAGE NUMBER, AND DOCUMENT LAYOUT NOTES:
%
% The next \usepackage line changes the layout for CV style section
% headings as marginal notes. It also sets up the paper size as either
% letter or A4. By default, letter was used. If A4 paper is desired,
% comment out the letterpaper lines and uncomment the a4paper lines.
%
% As you can see, the margin widths and section title widths can be
% easily adjusted.
%
% ALSO: Notice that the includefoot option can be commented OUT in order
% to put the PAGE NUMBER *IN* the bottom margin. This will make the
% effective text area larger.
%
% IF YOU WISH TO REMOVE THE ``of LASTPAGE'' next to each page number,
% see the note about the +LP and -LP lines below. Comment out the +LP
% and uncomment the -LP.
%
% IF YOU WISH TO REMOVE PAGE NUMBERS, be sure that the includefoot line
% is uncommented and ALSO uncomment the \pagestyle{empty} a few lines
% below.
%

%% Use these lines for letter-sized paper
\usepackage[paper=letterpaper,
%includefoot, % Uncomment to put page number above margin
marginparwidth=1.025in,     % Length of section titles
marginparsep=.05in,       % Space between titles and text
margin=1in,               % 1 inch margins
includemp]{geometry}

%% Use these lines for A4-sized paper
%\usepackage[paper=a4paper,
%            %includefoot, % Uncomment to put page number above margin
%            marginparwidth=30.5mm,    % Length of section titles
%            marginparsep=1.5mm,       % Space between titles and text
%            margin=25mm,              % 25mm margins
%            includemp]{geometry}

%% More layout: Get rid of indenting throughout entire document
\setlength{\parindent}{0in}

% Provides special list environments and macros to create new ones
\usepackage[shortlabels]{enumitem}

% Simpler bibsections for CV sections
% (thanks to natbib for inspiration)
%
% * For lists of references with hanging indents and no numbers:
%
%   \begin{bibsection}
%       \item ...
%   \end{bibsection}
%
% * For numbered lists of references (with hanging indents):
%
%   \begin{bibenum}
%       \item ...
%   \end{bibenum}
%
%   Note that bibenum numbers continuously throughout. To reset the
%   counter, use
%
%   \restartlist{bibenum}
%
%   at the place where you want the numbering to reset.

\makeatletter
\newlength{\bibhang}
\setlength{\bibhang}{1em}
\newlength{\bibsep}
{\@listi \global\bibsep\itemsep \global\advance\bibsep by\parsep}
\newlist{bibsection}{itemize}{3}
\setlist[bibsection]{label=,leftmargin=\bibhang,%
	itemindent=-\bibhang,
	itemsep=\bibsep,parsep=\z@,partopsep=0pt,
	topsep=0pt}
\newlist{bibenum}{enumerate}{3}
\setlist[bibenum]{label=[\arabic*],resume,leftmargin={\bibhang+\widthof{[999]}},%
	itemindent=-\bibhang,
	itemsep=\bibsep,parsep=\z@,partopsep=0pt,
	topsep=0pt}
\let\oldendbibenum\endbibenum
\def\endbibenum{\oldendbibenum\vspace{-.6\baselineskip}}
\let\oldendbibsection\endbibsection
\def\endbibsection{\oldendbibsection\vspace{-.6\baselineskip}}
\makeatother

%% Reference the last page in the page number
%
% NOTE: comment the +LP line and uncomment the -LP line to have page
%       numbers without the ``of ##'' last page reference)
%
% NOTE: uncomment the \pagestyle{empty} line to get rid of all page
%       numbers (make sure includefoot is commented out above)
%
\usepackage{fancyhdr,lastpage}
\pagestyle{fancy}
%\pagestyle{empty}      % Uncomment this to get rid of page numbers
\fancyhf{}\renewcommand{\headrulewidth}{0pt}
\fancyfootoffset{\marginparsep+\marginparwidth}
\newlength{\footpageshift}
\setlength{\footpageshift}
{0.5\textwidth+0.5\marginparsep+0.5\marginparwidth-2in}
\lfoot{\hspace{\footpageshift}%
	\parbox{4in}{\, \hfill %
		\arabic{page} of \protect\pageref*{LastPage} % +LP
		%                    \arabic{page}                               % -LP
		\hfill \,}}

% Finally, give us PDF bookmarks
\usepackage{xcolor,hyperref}
\definecolor{accent}{HTML}{10427A}
\hypersetup{colorlinks,breaklinks,
	linkcolor=accent,urlcolor=accent,
	anchorcolor=accent,citecolor=accent}

%%%%%%%%%%%%%%%%%%%%%%%% End Document Setup %%%%%%%%%%%%%%%%%%%%%%%%%%%%


%%%%%%%%%%%%%%%%%%%%%%%%%%% Helper Commands %%%%%%%%%%%%%%%%%%%%%%%%%%%%

%%% HEADING AT TOP OF CURRICULUM VITAE

% The title (name) with a horizontal rule under it
% (optional argument typesets an object right-justified across from name
%  as well)
%
% Usage: \makeheading{name}
%        OR
%        \makeheading[right_object]{name}
%
% Place at top of document. It should be the first thing.
% If ``right_object'' is provided in the square-braced optional
% argument, it will be right justified on the same line as ``name'' at
% the top of the CV. For example:
%
%       \makeheading[\emph{Curriculum vitae}]{Your Name}
%
% will put an emphasized ``Curriculum vitae'' at the top of the document
% as a title. Likewise, a picture could be included:
%
%   \makeheading[{\includegraphics[height=1.5in]{my_picture}}]{Your Name}
%
% the picture will be flush right across from the name. For this example
% to work, make sure the extra set of curly braces is included. Also
% makes ure that \usepackage{graphicx} is somewhere in the preamble.
\newcommand{\makeheading}[2][]%
{\hspace*{-\marginparsep minus \marginparwidth}%
	\begin{minipage}[t]{\textwidth+\marginparwidth+\marginparsep}%
		{\large \bfseries \scshape #2 \hfill #1}\\[-0.15\baselineskip]%
		\rule{\columnwidth}{1pt}%
\end{minipage}}

%%% SECTION HEADINGS

% The section headings. Flush left in small caps down pseudo-margin.
%
% Usage: \section{section name}
\renewcommand{\section}[1]{\pagebreak[3]%
	\vspace{0.5\baselineskip}%
	\phantomsection\addcontentsline{toc}{section}{#1}%
	\noindent\llap{\bf\scshape\smash{\parbox[t]{\marginparwidth}{\hyphenpenalty=10000\raggedright \textcolor{black}{#1}}}}%
	\vspace{-\baselineskip}\par}

\renewcommand{\subsection}[1]{\textcolor{black}{#1}}

%%% LISTS

% This macro alters a list by removing some of the space that follows the list
% (is used by lists below)
\newcommand*\fixendlist[1]{%
	\expandafter\let\csname preFixEndListend#1\expandafter\endcsname\csname end#1\endcsname
	\expandafter\def\csname end#1\endcsname{\csname preFixEndListend#1\endcsname\vspace{-0.6\baselineskip}}}

% These macros help ensure that items in outer-type lists do not get
% separated from the next line by a page break
% (they are used by lists below)
\let\originalItem\item
\newcommand*\fixouterlist[1]{%
	\expandafter\let\csname preFixOuterList#1\expandafter\endcsname\csname #1\endcsname
	\expandafter\def\csname #1\endcsname{\let\oldItem\item\def\item{\pagebreak[2]\oldItem}\csname preFixOuterList#1\endcsname}
	\expandafter\let\csname preFixOuterListend#1\expandafter\endcsname\csname end#1\endcsname
	\expandafter\def\csname end#1\endcsname{\let\item\oldItem\csname preFixOuterListend#1\endcsname}}
\newcommand*\fixinnerlist[1]{%
	\expandafter\let\csname preFixInnerList#1\expandafter\endcsname\csname #1\endcsname
	\expandafter\def\csname #1\endcsname{\let\oldItem\item\let\item\originalItem\csname preFixInnerList#1\endcsname}
	\expandafter\let\csname preFixInnerListend#1\expandafter\endcsname\csname end#1\endcsname
	\expandafter\def\csname end#1\endcsname{\csname preFixInnerListend#1\endcsname\let\item\oldItem}}

% An itemize-style list with lots of space between items
%
% Usage:
%   \begin{outerlist}
%       \item ...    % (or \item[] for no bullet)
%   \end{outerlist}
\newlist{outerlist}{itemize}{3}
\setlist[outerlist]{label=\enskip\textbullet,leftmargin=*}
\fixendlist{outerlist}
\fixouterlist{outerlist}

% An environment IDENTICAL to outerlist that has better pre-list spacing
% when used as the first thing in a \section
%
% Usage:
%   \begin{lonelist}
%       \item ...    % (or \item[] for no bullet)
%   \end{lonelist}
\newlist{lonelist}{itemize}{3}
\setlist[lonelist]{label=\enskip\textbullet,leftmargin=*,partopsep=0pt,topsep=0pt}
\fixendlist{lonelist}
\fixouterlist{lonelist}

% An itemize-style list with little space between items
%
% Usage:
%   \begin{innerlist}
%       \item ...    % (or \item[] for no bullet)
%   \end{innerlist}
\newlist{innerlist}{itemize}{3}
\setlist[innerlist]{label=\enskip--,leftmargin=*,parsep=0pt,itemsep=0pt,topsep=0pt,partopsep=0pt}
\fixinnerlist{innerlist}

% An environment IDENTICAL to innerlist that has better pre-list spacing
% when used as the first thing in a \section
%
% Usage:
%   \begin{loneinnerlist}
%       \item ...    % (or \item[] for no bullet)
%   \end{loneinnerlist}
\newlist{loneinnerlist}{itemize}{3}
\setlist[loneinnerlist]{label=\enskip\textbullet,leftmargin=*,parsep=0pt,itemsep=0pt,topsep=0pt,partopsep=0pt}
\fixendlist{loneinnerlist}
\fixinnerlist{loneinnerlist}

%%% EXTRA SPACE

% To add some paragraph space between lines.
% This also tells LaTeX to preferably break a page on one of these gaps
% if there is a needed pagebreak nearby.
\newcommand{\blankline}{\quad\pagebreak[3]}
\newcommand{\halfblankline}{\quad\vspace{-0.5\baselineskip}\pagebreak[3]}
\newcommand{\quarterblankline}{\quad\vspace{-0.75\baselineskip}\pagebreak[3]}
%%% FORMATTING MACROS

% Provides a linked \doi{#1} that links doi:#1 to http://dx.doi.org/#1
\usepackage{doi}
% To change the text before the DOI, adjust this command
%\renewcommand\doitext{doi:}

% Provides a linked \url{#1} that doesn't require escape characters
\usepackage{url}

% You can adjust the style \url{} uses here:
% (options are: same, rm, sf, tt; defaults to tt)
\urlstyle{same}

% For \email{ADDRESS}, links ADDRESS to the url mailto:ADDRESS
% (uncomment to typeset the e\-/mail address in typewriter font;
%  otherwise, will be typeset in the \urlstyle above)
%\DeclareUrlCommand\emaillink{\urlstyle{tt}}
\providecommand*\emaillink[1]{\nolinkurl{#1}}
\providecommand*\email[1]{\href{mailto:#1}{\emaillink{#1}}}
\providecommand*\ttlink[2]{\href{#1}{\texttt{\textcolor{accent}{#2}}}}
\providecommand*\titlelink[2]{\href{#1}{\textcolor{accent}{#2}}}

\providecommand\BibTeX{{B\kern-.05em{\sc i\kern-.025em b}\kern-.08em \TeX}}
\providecommand\Matlab{\textsc{Matlab}}

% Custom hyphenation rules for words that LaTeX has trouble with
\hyphenation{bio-mim-ic-ry bio-in-spi-ra-tion re-us-a-ble pro-vid-er Media-Wiki}

%%%%%%%%%%%%%%%%%%%%%%%% End Helper Commands %%%%%%%%%%%%%%%%%%%%%%%%%%%

%%%%%%%%%%%%%%%%%%%%%%%%% Begin CV Document %%%%%%%%%%%%%%%%%%%%%%%%%%%%

\begin{document}
	\makeheading{\huge{Aaron Mueller}}
	\\

	\section{Contact}
	Khoury College of Computer Sciences
	\hfill\href{mailto:aa.mueller@northeastern.edu}{\texttt{aa.mueller@northeastern.edu}} \\
	Northeastern University
	\hfill\ttlink{https://aaronmueller.github.io}{aaronmueller.github.io}\\
	177 Huntington Ave., 22 Fl.
	\hfill\ttlink{https://github.com/aaronmueller}{github.com/aaronmueller}\\
	Boston, MA 02115 (USA)
	%\hfill\texttt{+1 (502) 550-4938}
	
	%%
	%% In modern CV's, it seems like ``Objective'' is frowned upon. Instead,
	%% incorporate it into a well-constructed cover letter. The ``More
	%% information'' can go at the end of the CV, but it should not distract
	%% from the section giving references available to contact.
	%%
	%
	% \section{Objective}
	%
	% Placement in an academic position (i.e., faculty, postdoctoral, or
	% research scientist) that allows for advanced research in distributed
	% complex adaptive systems (i.e., modeling, analysis, design, and
	% verification) with a particular focus on the control of engineered
	% agents (e.g., for communications, control, software, electronics, and
	% sustainability) and the analysis of biological phenomena (e.g.,
	% self-organization, ecological rationality)
	% \begin{innerlist}
	% \item More information and auxiliary documents can be found at\\\url{http://www.tedpavlic.com/facjobsearch/}
	% \end{innerlist}

	
	\section{Research Interests}
	\begin{itemize}[topsep=0pt]
		\setlength{\itemindent}{-.2in}
		\itemsep-0.3em
		\item Natural language processing
		\item Robust generalization
		\item Mechanistic interpretability
		\item Computational psycholinguistics
	\end{itemize}
	
	\section{Education}

	{\textbf{Johns Hopkins University}}
	\hfill Baltimore, MD
	\begin{outerlist}[topsep=0pt]
	\itemsep-0.3em
	\item[] Ph.D., Computer Science
	\hfill Aug.\ 2023
	\item[] M.S.E., Computer Science
	\hfill May 2020
	\item[] GPA: 3.9/4.0
	\item[] \textit{Thesis}: Emergent Syntactic Behaviors and Mechanisms in Neural Language Models
	\item[] \textit{Advisors}: Tal Linzen, Mark Dredze\\
	\end{outerlist}

	{\textbf{New York University}}
	\hfill New York, NY 
	\begin{outerlist}[topsep=0pt]
	\itemsep-0.3em
	\item[] Visiting academic, Center for Data Science
	\hfill Aug.\ 2021 -- Aug.\ 2023
	\item[] \emph{Advisor}: Tal Linzen\\ 
	\end{outerlist}

	{\textbf{University of Kentucky}}
	\hfill Lexington, KY
	\begin{outerlist}[topsep=0pt]
		\itemsep-0.3em
		\item[] B.S., Computer Science. \emph{Honors}
		\hfill May 2018
		\item[] B.S., Linguistics. \emph{Honors}
		\hfill May 2018
		\item[] GPA: 4.0/4.0. \textit{Summa cum laude}
	\end{outerlist}

	\blankline

	\section{Academic Positions}

	{\textbf{Northeastern University}}
	\hfill Boston, MA\\
		\textit{Zuckerman Postdoctoral Fellow}, Khoury College of Computer Sciences
		\hfill Aug. 2023 -- Present\\
		\emph{Advisor}: David Bau

	\halfblankline

	{\textbf{Technion -- Israel Institute of Technology}}
	\hfill Haifa, Israel\\
		\textit{Zuckerman Postdoctoral Fellow}, Department of Computer Science
		\hfill Aug. 2023 -- Present\\
		\emph{Advisor}: Yonatan Belinkov
		
	\section{Industry Experience}

	{\textbf{Meta}}
	\hfill Menlo Park, CA\\
		\textit{Research Intern}
		\hfill May -- Nov.\ 2022\\
		\emph{Manager}: Kanika Narang
		\begin{innerlist}
			\item Research in retrieval-augmented generative models for few-shot question answering.
			\item Resulted in improved F\textsubscript{1} on multiple QA and classification datasets using far fewer parameters than state-of-the-art models. Also resulted in a publication at ACL [\ref{pub:metalearn}].
		\end{innerlist}
	
	\halfblankline
	
	{\textbf{Amazon Web Services (AWS)}}
	\hfill Santa Clara, CA\\
		\textit{Applied Scientist Intern}
		\hfill May -- Aug.\ 2021\\
		\emph{Manager}: Saab Mansour
		\begin{innerlist}
			\item Research in pre-training methods for improving goal-oriented dialogue agents.
			\item Resulted in state-of-the-art few-shot intent classification accuracy (>30\% 1-shot gains) and a publication at ACL [\ref{pub:lsap}].
		\end{innerlist}

	\halfblankline

	{\textbf{Raytheon BBN Technologies}}
	\hfill Cambridge, MA\\
		\textit{Research Intern}
		\hfill May -- Aug.\ 2019\\
		\emph{Manager}: Ilana Heintz
		\begin{innerlist}
			\item Research in low-resource cross-lingual word alignment and entity linking.
			\item Implemented convolutional neural machine translation models rivaling our prior seq2seq model's BLEU with over 20\% faster training and over 50\% faster inference.
		\end{innerlist}

	\section{Publications}

	\subsection{\textbf{Peer-reviewed Articles}}
	\begin{enumerate}[leftmargin=*, topsep=0pt, itemsep=0.25ex, partopsep=0ex, parsep=1ex]
	
	\item \textbf{Aaron Mueller}, Albert Webson, Jackson Petty, Tal Linzen. ``\titlelink{https://arxiv.org/abs/2311.07811}{In-context Learning Generalizes, But Not Always Robustly: The Case of Syntax}.'' To appear in \emph{Proceedings of the North American Chapter of the Association for Computational Linguistics (NAACL)}, 2024.\label{pub:icl-ood}
	
	\item Eric Todd, Millicent L. Li, Arnab Sen Sharma, \textbf{Aaron Mueller}, Byron C. Wallace, David Bau. ``\titlelink{https://arxiv.org/abs/2310.15213}{Function Vectors in Large Language Models}.'' In \emph{Proceedings of the International Conference on Learning Representations (ICLR)}, 2024.\label{pub:function-vectors}
	
	\item \textbf{Aaron Mueller}, Tal Linzen. ``\titlelink{https://arxiv.org/abs/2305.19905}{How to Plant Trees in Language Models: Data and Architectural Effects on the Emergence of Syntactic Inductive Biases}.'' In \emph{Proceedings of the Association for Computational Linguistics (ACL)}, 2023.
		
	\item \textbf{Aaron Mueller}, Kanika Narang, Lambert Mathias, Qifan Wang, Hamed Firooz. ``\titlelink{https://arxiv.org/abs/2307.00119}{Meta-training with Demonstration Retrieval for Efficient Few-shot Learning}.'' In \textit{Findings of the Association for Computational Linguistics (ACL)}, 2023.\label{pub:metalearn}
	
	\item Koustuv Sinha, Jon Gauthier, \textbf{Aaron Mueller}, Kanishka Misra, Keren Fuentes, Roger Levy, Adina Williams. ``\titlelink{https://arxiv.org/abs/2212.08979}{Language Model Acceptability Judgements Are Not Always Robust to Context}.'' In \emph{Proceedings of the Association for Computational Linguistics (ACL)}, 2023. \textbf{\textcolor{accent}{Outstanding Paper Award.}}
	
	\item Ian R. McKenzie, Alexander Lyzhov, Michael Martin Pieler, Alicia Parrish, \textbf{Aaron Mueller}, Ameya Prabhu, Euan McLean, Xudong Shen, Joe Cavanagh, Andrew George Gritsevskiy, Derik Kauffman, Aaron T. Kirtland, Zhengping Zhou, Yuhui Zhang, Sicong Huang, Daniel Wurgaft, Max Weiss, Alexis Ross, Gabriel Recchia, Alisa Liu, Jiacheng Liu, Tom Tseng, Tomasz Korbak, Najoung Kim, Samuel R. Bowman, Ethan Perez. ``\titlelink{https://arxiv.org/abs/2306.09479}{Inverse Scaling: When Bigger Isn't Better}.'' In \emph{Transactions on Machine Learning Research (TMLR)}, 2023. \textbf{\textcolor{accent}{Featured Paper.}}

	\item Julian Michael, Ari Holtzman, Alicia Parrish, \textbf{Aaron Mueller}, Alex Wang, Angelica Chen, Divyam Madaan, Nikita Nangia, Richard Yuanzhe Pang, Jason Phang, Samuel R.\ Bowman. ``\titlelink{https://arxiv.org/abs/2208.12852}{What Do NLP Researchers Believe? Results of the NLP Community Metasurvey}.'' In \emph{Proceedings of the Association for Computational Linguistics (ACL)}, 2023.
	
	\item \textbf{Aaron Mueller}, Robert Frank, Tal Linzen, Luheng Wang, Sebastian Schuster. ``\titlelink{https://aclanthology.org/2022.findings-acl.106.pdf}{Coloring the Blank Slate: Pre-training Imparts a Hierarchical Inductive Bias to Sequence-to-sequence Models}.'' In \emph{Findings of the Association for Computational Linguistics (ACL)}, 2022.

	\item \textbf{Aaron Mueller}, Jason Krone, Salvatore Romeo, Saab Mansour, Elman Mansimov, Yi Zhang, Dan Roth. ``\titlelink{https://aclanthology.org/2022.acl-long.570/}{Label Semantic Aware Pre-training for Few-shot Text Classification}.'' In \emph{Proceedings of the Association for Computational Linguistics (ACL)}, 2022.\label{pub:lsap}
	
	\item \textbf{Aaron Mueller}, Yu Xia, Tal Linzen. ``\titlelink{https://arxiv.org/abs/2210.14328}{Causal Analysis of Syntactic Agreement Neurons in Multilingual Language Models}.'' In \emph{Proceedings of the Conference on Computational Natural Language Learning (CoNLL)}, 2022.\label{pub:causal-multiling}
	
	\item Alexandra DeLucia, Shijie Wu, \textbf{Aaron Mueller}, Carlos Aguirre, Mark Dredze, Philip Resnik. ``\titlelink{https://preview.aclanthology.org/emnlp-22-ingestion/2022.emnlp-main.415/}{\textsc{Bernice}: A Multilingual Pre-trained Encoder for Twitter}.'' In \emph{Proceedings of the Conference on Empirical Methods in Natural Language Processing (EMNLP)}, 2022.
			
	\item \textbf{Aaron Mueller}, Mark Dredze. ``\titlelink{https://aclanthology.org/2021.naacl-main.243/}{Fine-tuning Encoders for Improved Monolingual and Zero-shot Polylingual Neural Topic Modeling}.'' In \emph{Proceedings of the North American Chapter of the Association for Computational Linguistics (NAACL)}, 2021.  
	
	\item \textbf{Aaron Mueller}, Zach Wood-Doughty, Silvio Amir, Mark Dredze, Alicia L.\ Nobles. ``\titlelink{https://dl.acm.org/doi/10.1145/3449181}{Demographic Representation and Collective Storytelling in the Me Too Twitter Hashtag Activism Movement.}'' In \emph{Proceedings of the Association for Computing Machinery (ACM) on Human-Computer Interaction (HCI), vol.\ CSCW1}, 2021.

	\item Matthew Finlayson*, \textbf{Aaron Mueller}*, Sebastian Gehrmann, Stuart Shieber, Tal Linzen, Yonatan Belinkov. ``\titlelink{https://aclanthology.org/2021.acl-long.144/}{Causal Analysis of Syntactic Agreement Mechanisms in Neural Language Models}.'' In \emph{Proceedings of the Association for Computational Linguistics (ACL)}, 2021. [*Equal contribution]\label{pub:causal}
	
	\item Alexandra DeLucia*, \textbf{Aaron Mueller}*, Xiang Lisa Li, João Sedoc. ``\titlelink{https://aclanthology.org/2021.gem-1.16/}{Decoding Methods for Neural Narrative Generation}.'' In \emph{Proceedings of the Workshop on Generation Evaluation and Metrics (GEM) at Association for Computational Linguistics (ACL)}, 2021. [*Equal contribution]

	\item \textbf{Aaron Mueller}, Garrett Nicolai, Panayiota Petrou-Zeniou, Natalia Talmina, Tal Linzen. ``\titlelink{https://aclanthology.org/2020.acl-main.490/}{Cross-linguistic Syntactic Evaluation of Word Prediction Models}.'' In \emph{Proceedings of the Association for Computational Linguistics (ACL)}, 2020.

	\item \textbf{Aaron Mueller}, Garrett Nicolai, Arya D. McCarthy, Dylan Lewis, Winston Wu, David Yarowsky. ``\titlelink{https://aclanthology.org/2020.lrec-1.458/}{An Analysis of Massively Multilingual Neural Machine Translation for Low-Resource Languages}.'' In \emph{Proceedings of the Language Resources and Evaluation Conference (LREC)}, 2020.

	\item Arya D. McCarthy, Rachel Wicks, Dylan Lewis, \textbf{Aaron Mueller}, Winston Wu, Oliver Adams, Garrett Nicolai, Matt Post, David Yarowsky. ``\titlelink{https://aclanthology.org/2020.lrec-1.352/}{The Johns Hopkins University Bible Corpus: 1600+ Tongues for Typological Exploration}.'' In \emph{Proceedings of the Language Resources and Evaluation Conference (LREC)}, 2020.

	\item Garrett Nicolai, Dylan Lewis, Arya D. McCarthy, \textbf{Aaron Mueller}, Winston Wu, David Yarowsky. ``\titlelink{https://aclanthology.org/2020.lrec-1.488/}{Fine-grained Morphosyntactic Analysis and Generation Tools for More Than One Thousand Languages}.'' In \emph{Proceedings of the Language Resources and Evaluation Conference (LREC)}, 2020.

	\item Marten van Schijndel, \textbf{Aaron Mueller}, Tal Linzen. ``\titlelink{https://aclanthology.org/D19-1592/}{Quantity Doesn't Buy Quality Syntax with Neural Language Models}.'' In \emph{Proceedings of the Conference on Empirical Methods in Natural Language Processing (EMNLP)}, 2019.

	\item Arya D. McCarthy, Winston Wu, \textbf{Aaron Mueller}, Bill Watson, David Yarowsky. ``\titlelink{https://aclanthology.org/D19-1229/}{Modeling Color Terminology Across Thousands of Languages}.'' In \emph{Proceedings of the Conference on Empirical Methods in Natural Language Processing (EMNLP)}, 2019.

	\item \textbf{Aaron Mueller}*, Yash Kumar Lal*. ``\titlelink{https://aclanthology.org/W19-6807/}{Sentence-Level Adaptation for Low-Resource Neural Machine Translation}.'' In \emph{Proceedings of the Workshop on Technologies for Machine Translation of Low-Resource Languages (LoResMT) at Machine Translation Summit (MTSummit)}, 2019. [*Equal contribution]\label{pub:sent-level}
	\end{enumerate}

	\halfblankline

	\subsection{\textbf{Preprints \& In Submission}}
	\begin{enumerate}[resume, leftmargin=*, topsep=0pt, itemsep=0.25ex, partopsep=0ex, parsep=1ex]

	\item Samuel Marks, Can Rager, Eric J. Michaud, Yonatan Belinkov, David Bau, \textbf{Aaron Mueller}. ``\titlelink{https://arxiv.org/abs/2403.19647}{Sparse Feature Circuits: Discovering and Editing Interpretable Causal Graphs in Language Models}.'' arXiv preprint, 2024.\label{pub:feature-circuits}

	\end{enumerate}

	\halfblankline

	\subsection{\textbf{Invited Publications}}
	
	\begin{enumerate}[resume, leftmargin=*, topsep=0pt, itemsep=0.25ex, partopsep=0ex, parsep=1ex]
	\item Alex Warstadt*, \textbf{Aaron Mueller}*, Leshem Choshen, Ethan Wilcox, Juan Ciro, Rafael Mosquera, Bhargavi Paranjabe, Adina Williams, Tal Linzen, Ryan Cotterell. ``\titlelink{https://arxiv.org/abs/2301.11796}{Findings of the BabyLM Challenge: Sample-efficient Pretraining on Developmentally Plausible Corpora}.'' Shared task proceedings in \emph{Conference on Computational Natural Language Learning (CoNLL)}, 2023. [*Equal contribution]
	\end{enumerate}

	% \subsection{\textbf{Preprints \& In Submission}}
	% \begin{enumerate}[resume, leftmargin=*, topsep=0pt, itemsep=0.25ex, partopsep=0ex, parsep=1ex]	
				
	% \end{enumerate}

	\halfblankline
	
	\subsection{\textbf{Other Articles}}

	\begin{enumerate}[resume, leftmargin=*, topsep=0pt, itemsep=0.25ex, partopsep=0ex, parsep=1ex]
	\item \textbf{Aaron Mueller}. ``Emergent Syntactic Behaviors and Mechanisms in Neural Language \mbox{Models}.'' Ph.D.\ Dissertation, Johns Hopkins University, 2023. Committee: Tal Linzen, Mark Dredze, David Yarowsky, Yonatan Belinkov.
	\end{enumerate}

	\section{Invited\\Talks}
	\emph{Sparse Feature Circuits: Discovering and Editing Interpretable Causal Graphs\\in Language Models.}\\NLP Seminar, University of California, Santa Barbara (Santa Barbara, CA). Apr. 24, 2024.

	\halfblankline

	\emph{Evaluating and Surgically Improving Generalization in Language Models.}\\Responsible AI Seminar Series, Nokia Bell Labs (Cambridge, UK). Mar. 18, 2024.

	\halfblankline

	\emph{Evaluating and Surgically Improving Generalization in Language Models.}\\NLP Seminar, University of Pittsburgh (Pittsburgh, PA). Feb. 29, 2024.

	\halfblankline

	\emph{Evaluating and Surgically Improving Generalization in Language Models.}\\Deep Learning Superlab, Brown University (Providence, RI). Feb. 15, 2024.

	\halfblankline

	\emph{Planting Trees in Language Models: Emergent Syntactic Behaviors and Mechanisms\\from Pre-training.}\\Koller Lab, Saarland University (Saarbrücken, Germany). Feb. 7, 2023.

	\halfblankline

	\emph{Planting Trees in Language Models: Emergent Syntactic Behaviors and Mechanisms\\from Pre-training.}\\NLP Seminar, Technion -- Israel Institute of Technology (Haifa, Israel). Dec.\ 14, 2022.

	\halfblankline
	
	\emph{Planting Trees in Language Models: Emergent Syntactic Behaviors and Mechanisms\\from Pre-training.}\\Bar-Ilan NLP Seminar, Bar-Ilan University (Ramat Gan, Israel). Dec.\ 13, 2022.

	\halfblankline

	\emph{What Generalizations do Sequence-to-sequence Models Learn from Multilingual Text? Insights from Translation and Syntactic Transformations.}\\Multilingual Text Processing Group, National Research Council of Canada (Ottawa, ON). Mar.\ 4, 2022.
	
	\halfblankline

	\emph{Syntactic Agreement in Neural Language Models: How Well and Where Do They Perform Subject-Verb Agreement?}\\Language \& Understanding Group, Mila -- Québec Artificial Intelligence Institute (Montréal, QC). Mar.\ 22, 2021.
	
	\halfblankline

	\emph{Causal Mediation Analysis for Analyzing Neural Networks.}\\Fairness \& Interpretability Research Talk Series, Google (New York, NY). Mar.\ 17, 2021.
	
	\halfblankline

	\emph{Causal Analysis of Syntactic Agreement Mechanisms in Neural Language Models.}\\Center for Language \& Speech Processing Seminar, Johns Hopkins University (Baltimore, MD). Feb.\ 12, 2021.
	
	\halfblankline
	%\halfblankline

	%\section{Academic Research Experience}

	%{\textbf{Johns Hopkins University}}
	%\hfill Baltimore, MD\\
	%	\textit{Research Assistant}, Center for Language \& Speech Processing
	%	\hfill August 2018 -- Present \\
	%	Advisors: David Yarowsky, Mark Dredze
	%	\begin{innerlist}
	%		\item Research in neural machine translation, natural language generation, and massively multilingual NLP.
	%	\end{innerlist}

	%	\halfblankline

	%	\textit{Research Assistant}, Center for Language \& Speech Processing
	%	\hfill May 2016 -- August 2016\\
	%	Advisor: David Yarowsky
	%	\begin{innerlist}
	%		\item Research in statistical machine translation for low-resource languages.
	%		\item Implemented a lemma-based English-Uyghur translation model, built a morphological generator for Crimean Tatar, and created parsers to extract translation tables from foreign editions of Wiktionary.
	%	\end{innerlist}
	
	%\halfblankline

	%{\textbf{New York University}}
	%\hfill New York, NY\\
	%	\textit{Visiting Academic}, Center for Data Science
	%	\hfill August 2021 -- Present \\
	%	Advisor: Tal Linzen
	%	\begin{innerlist}
	%		\item Behavioral and causal probing into the (morpho)syntactic representations and abilities of neural language models and sequence-to-sequence models.
	%	\end{innerlist}

	%\halfblankline

%	{\textbf{University of Kentucky}}
%	\hfill Lexington, KY\\
%		\textit{Research Assistant}, Institutional Research \& Advanced Analytics Team
%		\hfill May 2018 -- August 2018\\
%		Advisors: Nathan Jacobs, Craig Rudick
%		\begin{innerlist}
%			\item Implemented deep LSTMs to learn student and course profiles for grade prediction.
%			\item Designed various ordinal loss functions.
%		\end{innerlist}

%	\halfblankline

	%{\textbf{University of Massachusetts Amherst}}
	%\hfill Amherst, MA\\
	%	\textit{Research Assistant}, Statistical Social Language Analysis Lab
	%	\hfill May 2017 -- August 2017\\
	%	Advisor: Brendan O'Connor
	%	\begin{innerlist}
	%		\item Research in entity-event extraction.
	%		\item Integrated entity mention and relation systems into pre-existing sentence-level entity-event extraction model. Also integrated document-level information (e.g., noun coreferences) as features.
	%	\end{innerlist}

	%\halfblankline

	\section{Fellowships and Awards}

	\textbf{Zuckerman Fellow}, \textit{International}
	\hfill 2023--2025\\
		Two-year postdoctoral fellowship. Supports research joint with an Israeli university and an American university. (\$126,000)

	\halfblankline

	\textbf{Microsoft Accelerate Foundation Models Research Award}, \textit{International}
	\hfill 2023\\
		Awarded for research on the capabilities of large language models. Provides OpenAI API credits and priority GPT-4 access. (\$10,000) 

	\halfblankline
	
	\textbf{National Science Foundation Graduate Research Fellow}, \textit{National}
	\hfill 2018 - 2023\\
		Five-year graduate research fellowship. Provides three years of Ph.D. funding. (\$135,000)
	
	\halfblankline
	
	\textbf{Gaines Fellow}, \textit{University of Kentucky}
	\hfill 2016 - 2018\\
		Two-year fellowship. Requires the completion of a juried project, a thesis project, and a seminar in the humanities. (\$5,000)
	
	\halfblankline

	\textbf{Patterson Scholar}, \textit{University of Kentucky}
	\hfill 2014 - 2018\\
		Four-year scholarship covering tuition, educational materials, and room \& board. Awarded to undergraduates who have earned National Merit semifinalist standing or higher. (\$86,000)

	\halfblankline

	\textbf{Goldwater Scholarship} (Honorable Mention), \textit{National}
	\hfill 2017
	
	\halfblankline
	
	\textbf{Phi Beta Kappa}, \textit{National}
	\hfill 2017

	\halfblankline
	
	\textbf{Raymond F. Betts Scholar}, \textit{University of Kentucky}
	\hfill 2017\\
		Awarded for thesis research. Used funds to design language technologies for low-resource
		dialects of French. (\$2,500)

	\halfblankline

	\textbf{Linguistics Research Award}, \textit{University of Kentucky}
	\hfill 2016\\
		Awarded to an undergraduate to facilitate a year-long
		research project in linguistics. (\$500)

	%\halfblankline
	
%	\textbf{National Merit Semifinalist}
%	\hfill 2014

	\section{Mentoring}
	\subsection{Ph.D.\ students}:
	\begin{innerlist}
	\item Aruna Sankaranarayanan (MIT). Joint with Forrest Davis. Research on \hfill 2023--Present\\
		natural and artificial grammar learning in language models.
	\item Eric W.\ Todd (Northeastern). Joint with David Bau. Research on how\hfill 2023--Present\\functions are
		represented in neural language models. Resulted in a\\publication at ICLR [\ref{pub:function-vectors}].
	\item Juan Diego Rodriguez (UT Austin). Joint with Kanishka Misra. Research\hfill 2023--Present\\
		  in how concepts are organized in language models.
	\end{innerlist}

	\halfblankline\vspace{-0.5pt}

	\subsection{Master's students}:
	\begin{innerlist}
	\item Dan Pechi (NYU). Research on imparting better inductive biases to\hfill2023\\language models.
	\item Swapnil Sharma (NYU). Research on evaluating summarization models.\hfill 2022--2023
	\item Yash Kumar Lal (JHU). Resulted in a workshop publication [\ref{pub:sent-level}].\hfill 2018--2019
	\end{innerlist}

	\halfblankline\vspace{-0.5pt}

	\subsection{Undergraduate researchers}:
	\begin{innerlist}
	\item Yu Xia (NYU). Resulted in a publication at CoNLL [\ref{pub:causal-multiling}].\hfill 2021--2022
	\item Matthew Finlayson (Harvard). Resulted in a publication at ACL [\ref{pub:causal}].\hfill 2020--2021
	\end{innerlist}

	\section{Teaching}

	{\textbf{Johns Hopkins University}}\\
	\textit{Teaching Assistant}\\
	\textit{Instructor}: Mathias Unberath
	\begin{innerlist}
		\item Machine Learning: AI System Design \& Development \hfill Spring 2020
	\end{innerlist}

	\halfblankline

	{\textbf{New York University}}\\
	\textit{Guest Lecture}\\
	\textit{Instructor}: Tal Linzen
	\begin{innerlist}
		\item Computational Linguistics \& Cognitive Science \hfill Spring 2023
	\end{innerlist}

	% \newpage
%	\section{Volunteer Service}
	
%	\textbf{Amnesty International}
%	\hfill 2012 - Present
%	\begin{innerlist}
%		\item Member of Amnesty International at Johns Hopkins and Washington, D.C.
%		\item \textit{Founder} and former \textit{Co-President} of Amnesty International at the University of Kentucky.
%		\item Organize fundraisers and informational events: concerts featuring local bands, panel discussions on contemporary issues, faculty debates, petitions, protests, and more.
%		\item Attendee at Amnesty's national Human Rights Conference.
%	\end{innerlist}
	
%	\halfblankline
	
%	\textbf{Oxfam International}
%	\hfill 2014 - Present
%	\begin{innerlist}
%		\item International organization dedicated to poverty eradication, disaster relief, and political advocacy for those facing harsh living conditions and injustice
%		worldwide.
%		\item Member of Oxfam in Washington, D.C.
%		\item \textit{Founder} and former \textit{Co-Coordinator} of Oxfam at the University of Kentucky.
%		\item Plan awareness-raising, fundraising, and petitioning events for Oxfam's GROW campaign with a distinct focus on food inequality in impoverished and war-stricken countries.
%	\end{innerlist}

%	\halfblankline
	
%	\textbf{Society for the Promotion of Undergraduate Research}
%	\hfill 2014 - 2017
%	\begin{innerlist}
%		\item University-of-Kentucky-based organization founded to assist undergraduates in finding research opportunities and in bettering their research.
%		\item \textit{Co-President} and former \textit{Director of Events}.
%		\item Plan and execute the Showcase of Undergraduate Scholars.
%		\item Plan research workshops to help undergraduates improve their abstracts, presentations, and papers.
%		\item Organize faculty panel discussions, promotional events, and information sessions on getting involved in undergraduate research and on scholarship and summer research opportunities.
%		\item Merged with the Student Outreach Team in May 2017 (joined upon merge).
%	\end{innerlist}

	\section{Service}
	\subsection{Organizing Committees}:
	\begin{innerlist}[itemsep=0pt]
	\item \titlelink{https://babylm.github.io}{The BabyLM Challenge} (2023, 2024)
	\item \titlelink{https://github.com/inverse-scaling/prize}{The Inverse Scaling Prize} (2022)
	\end{innerlist}

	\halfblankline

	\subsection{Reviewing}:
	\begin{innerlist}
	\item ACL Rolling Review (Oct.\ 2021 -- present; monthly)
	\item NAACL (2024, 2021)
	\item ACL (2023, 2022, 2020)
	\item EMNLP (2023, 2022, 2019)
	\item CoNLL (2023, 2022)
	\item FAccT (2024)
	\item TACL (2022)
	\item CSCW (2021)
	\item COLING (2020)
	\end{innerlist}
	\halfblankline

	\section{Skills}
	
	\subsection{Programming}:
	%
	\begin{innerlist}
		\item Languages: Python, C$++$, HTML, CSS, Javascript, Bash
		\item Machine Learning Toolkits: PyTorch (incl.\ HuggingFace, fairseq, sockeye), NLTK, Scikit-learn, numpy
		\item Version Control: DVCS (Git, Bitbucket)
	\end{innerlist}
	
	\halfblankline
	
	\subsection{Linguistic Tools}:
	\begin{innerlist}
		\item Praat, AntConc, QGIS, Audacity
	\end{innerlist}

	\section{Natural Languages}
	English (native language). French (B2, Canadian).
	
\end{document}

%%%%%%%%%%%%%%%%%%%%%%%%%% End CV Document %%%%%%%%%%%%%%%%%%%%%%%%%%%%%

%----------------------------------------------------------------------%
% The following is copyright and licensing information for
% redistribution of this LaTeX source code; it also includes a liability
% statement. If this source code is not being redistributed to others,
% it may be omitted. It has no effect on the function of the above code.
%----------------------------------------------------------------------%
% Copyright (c) 2007, 2008, 2009, 2010, 2011 by Theodore P. Pavlic
%
% Unless otherwise expressly stated, this work is licensed under the
% Creative Commons Attribution-Noncommercial 3.0 United States License. To
% view a copy of this license, visit
% http://creativecommons.org/licenses/by-nc/3.0/us/ or send a letter to
% Creative Commons, 171 Second Street, Suite 300, San Francisco,
% California, 94105, USA.
%
% THE SOFTWARE IS PROVIDED "AS IS", WITHOUT WARRANTY OF ANY KIND, EXPRESS
% OR IMPLIED, INCLUDING BUT NOT LIMITED TO THE WARRANTIES OF
% MERCHANTABILITY, FITNESS FOR A PARTICULAR PURPOSE AND NONINFRINGEMENT.
% IN NO EVENT SHALL THE AUTHORS OR COPYRIGHT HOLDERS BE LIABLE FOR ANY
% CLAIM, DAMAGES OR OTHER LIABILITY, WHETHER IN AN ACTION OF CONTRACT,
% TORT OR OTHERWISE, ARISING FROM, OUT OF OR IN CONNECTION WITH THE
% SOFTWARE OR THE USE OR OTHER DEALINGS IN THE SOFTWARE.