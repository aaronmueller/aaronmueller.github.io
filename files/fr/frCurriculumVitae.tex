%%%%%%%%%%%%%%%%%%%%%%%%%%%%%%%%%%%%%%%%%%%%%%%%%%%%%%%%%%%%%%%%%%%%%%%%
%%%%%%%%%%%%%%%%%%%%%% Simple LaTeX CV Template %%%%%%%%%%%%%%%%%%%%%%%%
%%%%%%%%%%%%%%%%%%%%%%%%%%%%%%%%%%%%%%%%%%%%%%%%%%%%%%%%%%%%%%%%%%%%%%%%

%%%%%%%%%%%%%%%%%%%%%%%%%%%%%%%%%%%%%%%%%%%%%%%%%%%%%%%%%%%%%%%%%%%%%%%%
%% NOTE: If you find that it says                                     %%
%%                                                                    %%
%%                           1 of ??                                  %%
%%                                                                    %%
%% at the bottom of your first page, this means that the AUX file     %%
%% was not available when you ran LaTeX on this source. Simply RERUN  %%
%% LaTeX to get the ``??'' replaced with the number of the last page  %%
%% of the document. The AUX file will be generated on the first run   %%
%% of LaTeX and used on the second run to fill in all of the          %%
%% references.                                                        %%
%%%%%%%%%%%%%%%%%%%%%%%%%%%%%%%%%%%%%%%%%%%%%%%%%%%%%%%%%%%%%%%%%%%%%%%%

%%%%%%%%%%%%%%%%%%%%%%%%%%%% Document Setup %%%%%%%%%%%%%%%%%%%%%%%%%%%%

% Don't like 10pt? Try 11pt or 12pt
\documentclass[10pt]{article}
\RequirePackage[T1]{fontenc}
\usepackage[utf8]{inputenc}

% LaTeX will typeset using Computer Modern Roman, which a lot of
% non-mathematicians and non-engineers won't like. Also, a few PDF
% viewers may not render CMR very well. Instead, Times New Roman can
% be used. That's what this package does.
\usepackage{times}

% The automated optical recognition software used to digitize resume
% information works best with fonts that do not have serifs. This
% command uses a sans serif font throughout. Uncomment both lines (or at
% least the second) to restore a Roman font (i.e., a font with serifs).
% (NOTE: This requires the times package above)
%\renewcommand{\familydefault}{\sfdefault}

% This is a helpful package that puts math inside length specifications
\usepackage{calc}

% This package helps LaTeX auto-hyphenate hyphenated words if you use
% special hyphens. For example, bio\-/mimicry will properly hyphenate
% ``mimicry'' if necessary.
\usepackage[shortcuts]{extdash}

% Layout: Puts the section titles on left side of page
\reversemarginpar

%
%         PAPER SIZE, PAGE NUMBER, AND DOCUMENT LAYOUT NOTES:
%
% The next \usepackage line changes the layout for CV style section
% headings as marginal notes. It also sets up the paper size as either
% letter or A4. By default, letter was used. If A4 paper is desired,
% comment out the letterpaper lines and uncomment the a4paper lines.
%
% As you can see, the margin widths and section title widths can be
% easily adjusted.
%
% ALSO: Notice that the includefoot option can be commented OUT in order
% to put the PAGE NUMBER *IN* the bottom margin. This will make the
% effective text area larger.
%
% IF YOU WISH TO REMOVE THE ``of LASTPAGE'' next to each page number,
% see the note about the +LP and -LP lines below. Comment out the +LP
% and uncomment the -LP.
%
% IF YOU WISH TO REMOVE PAGE NUMBERS, be sure that the includefoot line
% is uncommented and ALSO uncomment the \pagestyle{empty} a few lines
% below.
%

%% Use these lines for letter-sized paper
\usepackage[paper=letterpaper,
%includefoot, % Uncomment to put page number above margin
marginparwidth=1.0in,     % Length of section titles
marginparsep=.05in,       % Space between titles and text
margin=0.9in,               % 1 inch margins
includemp]{geometry}

%% Use these lines for A4-sized paper
%\usepackage[paper=a4paper,
%            %includefoot, % Uncomment to put page number above margin
%            marginparwidth=30.5mm,    % Length of section titles
%            marginparsep=1.5mm,       % Space between titles and text
%            margin=25mm,              % 25mm margins
%            includemp]{geometry}

%% More layout: Get rid of indenting throughout entire document
\setlength{\parindent}{0in}

% Provides special list environments and macros to create new ones
\usepackage[shortlabels]{enumitem}

% Simpler bibsections for CV sections
% (thanks to natbib for inspiration)
%
% * For lists of references with hanging indents and no numbers:
%
%   \begin{bibsection}
%       \item ...
%   \end{bibsection}
%
% * For numbered lists of references (with hanging indents):
%
%   \begin{bibenum}
%       \item ...
%   \end{bibenum}
%
%   Note that bibenum numbers continuously throughout. To reset the
%   counter, use
%
%   \restartlist{bibenum}
%
%   at the place where you want the numbering to reset.

\makeatletter
\newlength{\bibhang}
\setlength{\bibhang}{1em}
\newlength{\bibsep}
{\@listi \global\bibsep\itemsep \global\advance\bibsep by\parsep}
\newlist{bibsection}{itemize}{3}
\setlist[bibsection]{label=,leftmargin=\bibhang,%
	itemindent=-\bibhang,
	itemsep=\bibsep,parsep=\z@,partopsep=0pt,
	topsep=0pt}
\newlist{bibenum}{enumerate}{3}
\setlist[bibenum]{label=[\arabic*],resume,leftmargin={\bibhang+\widthof{[999]}},%
	itemindent=-\bibhang,
	itemsep=\bibsep,parsep=\z@,partopsep=0pt,
	topsep=0pt}
\let\oldendbibenum\endbibenum
\def\endbibenum{\oldendbibenum\vspace{-.6\baselineskip}}
\let\oldendbibsection\endbibsection
\def\endbibsection{\oldendbibsection\vspace{-.6\baselineskip}}
\makeatother

%% Reference the last page in the page number
%
% NOTE: comment the +LP line and uncomment the -LP line to have page
%       numbers without the ``of ##'' last page reference)
%
% NOTE: uncomment the \pagestyle{empty} line to get rid of all page
%       numbers (make sure includefoot is commented out above)
%
\usepackage{fancyhdr,lastpage}
\pagestyle{fancy}
%\pagestyle{empty}      % Uncomment this to get rid of page numbers
\fancyhf{}\renewcommand{\headrulewidth}{0pt}
\fancyfootoffset{\marginparsep+\marginparwidth}
\newlength{\footpageshift}
\setlength{\footpageshift}
{0.5\textwidth+0.5\marginparsep+0.5\marginparwidth-2in}
\lfoot{\hspace{\footpageshift}%
	\parbox{4in}{\, \hfill %
		\arabic{page} of \protect\pageref*{LastPage} % +LP
		%                    \arabic{page}                               % -LP
		\hfill \,}}

% Finally, give us PDF bookmarks
\usepackage{color,hyperref}
\definecolor{darkblue}{rgb}{0.0,0.0,0.3}
\hypersetup{colorlinks,breaklinks,
	linkcolor=darkblue,urlcolor=darkblue,
	anchorcolor=darkblue,citecolor=darkblue}

%%%%%%%%%%%%%%%%%%%%%%%% End Document Setup %%%%%%%%%%%%%%%%%%%%%%%%%%%%


%%%%%%%%%%%%%%%%%%%%%%%%%%% Helper Commands %%%%%%%%%%%%%%%%%%%%%%%%%%%%

%%% HEADING AT TOP OF CURRICULUM VITAE

% The title (name) with a horizontal rule under it
% (optional argument typesets an object right-justified across from name
%  as well)
%
% Usage: \makeheading{name}
%        OR
%        \makeheading[right_object]{name}
%
% Place at top of document. It should be the first thing.
% If ``right_object'' is provided in the square-braced optional
% argument, it will be right justified on the same line as ``name'' at
% the top of the CV. For example:
%
%       \makeheading[\emph{Curriculum vitae}]{Your Name}
%
% will put an emphasized ``Curriculum vitae'' at the top of the document
% as a title. Likewise, a picture could be included:
%
%   \makeheading[{\includegraphics[height=1.5in]{my_picture}}]{Your Name}
%
% the picture will be flush right across from the name. For this example
% to work, make sure the extra set of curly braces is included. Also
% makes ure that \usepackage{graphicx} is somewhere in the preamble.
\newcommand{\makeheading}[2][]%
{\hspace*{-\marginparsep minus \marginparwidth}%
	\begin{minipage}[t]{\textwidth+\marginparwidth+\marginparsep}%
		{\large \bfseries #2 \hfill #1}\\[-0.15\baselineskip]%
		\rule{\columnwidth}{1pt}%
\end{minipage}}

%%% SECTION HEADINGS

% The section headings. Flush left in small caps down pseudo-margin.
%
% Usage: \section{section name}
\renewcommand{\section}[1]{\pagebreak[3]%
	\vspace{0.5\baselineskip}%
	\phantomsection\addcontentsline{toc}{section}{#1}%
	\noindent\llap{\scshape\smash{\parbox[t]{\marginparwidth}{\hyphenpenalty=10000\raggedright #1}}}%
	\vspace{-\baselineskip}\par}

%%% LISTS

% This macro alters a list by removing some of the space that follows the list
% (is used by lists below)
\newcommand*\fixendlist[1]{%
	\expandafter\let\csname preFixEndListend#1\expandafter\endcsname\csname end#1\endcsname
	\expandafter\def\csname end#1\endcsname{\csname preFixEndListend#1\endcsname\vspace{-0.6\baselineskip}}}

% These macros help ensure that items in outer-type lists do not get
% separated from the next line by a page break
% (they are used by lists below)
\let\originalItem\item
\newcommand*\fixouterlist[1]{%
	\expandafter\let\csname preFixOuterList#1\expandafter\endcsname\csname #1\endcsname
	\expandafter\def\csname #1\endcsname{\let\oldItem\item\def\item{\pagebreak[2]\oldItem}\csname preFixOuterList#1\endcsname}
	\expandafter\let\csname preFixOuterListend#1\expandafter\endcsname\csname end#1\endcsname
	\expandafter\def\csname end#1\endcsname{\let\item\oldItem\csname preFixOuterListend#1\endcsname}}
\newcommand*\fixinnerlist[1]{%
	\expandafter\let\csname preFixInnerList#1\expandafter\endcsname\csname #1\endcsname
	\expandafter\def\csname #1\endcsname{\let\oldItem\item\let\item\originalItem\csname preFixInnerList#1\endcsname}
	\expandafter\let\csname preFixInnerListend#1\expandafter\endcsname\csname end#1\endcsname
	\expandafter\def\csname end#1\endcsname{\csname preFixInnerListend#1\endcsname\let\item\oldItem}}

% An itemize-style list with lots of space between items
%
% Usage:
%   \begin{outerlist}
%       \item ...    % (or \item[] for no bullet)
%   \end{outerlist}
\newlist{outerlist}{itemize}{3}
\setlist[outerlist]{label=\enskip\textbullet,leftmargin=*}
\fixendlist{outerlist}
\fixouterlist{outerlist}

% An environment IDENTICAL to outerlist that has better pre-list spacing
% when used as the first thing in a \section
%
% Usage:
%   \begin{lonelist}
%       \item ...    % (or \item[] for no bullet)
%   \end{lonelist}
\newlist{lonelist}{itemize}{3}
\setlist[lonelist]{label=\enskip\textbullet,leftmargin=*,partopsep=0pt,topsep=0pt}
\fixendlist{lonelist}
\fixouterlist{lonelist}

% An itemize-style list with little space between items
%
% Usage:
%   \begin{innerlist}
%       \item ...    % (or \item[] for no bullet)
%   \end{innerlist}
\newlist{innerlist}{itemize}{3}
\setlist[innerlist]{label=\enskip--,leftmargin=*,parsep=0pt,itemsep=0pt,topsep=0pt,partopsep=0pt}
\fixinnerlist{innerlist}

% An environment IDENTICAL to innerlist that has better pre-list spacing
% when used as the first thing in a \section
%
% Usage:
%   \begin{loneinnerlist}
%       \item ...    % (or \item[] for no bullet)
%   \end{loneinnerlist}
\newlist{loneinnerlist}{itemize}{3}
\setlist[loneinnerlist]{label=\enskip\textbullet,leftmargin=*,parsep=0pt,itemsep=0pt,topsep=0pt,partopsep=0pt}
\fixendlist{loneinnerlist}
\fixinnerlist{loneinnerlist}

%%% EXTRA SPACE

% To add some paragraph space between lines.
% This also tells LaTeX to preferably break a page on one of these gaps
% if there is a needed pagebreak nearby.
\newcommand{\blankline}{\quad\pagebreak[3]}
\newcommand{\halfblankline}{\quad\vspace{-0.5\baselineskip}\pagebreak[3]}

%%% FORMATTING MACROS

% Provides a linked \doi{#1} that links doi:#1 to http://dx.doi.org/#1
\usepackage{doi}
% To change the text before the DOI, adjust this command
%\renewcommand\doitext{doi:}

% Provides a linked \url{#1} that doesn't require escape characters
\usepackage{url}

% You can adjust the style \url{} uses here:
% (options are: same, rm, sf, tt; defaults to tt)
\urlstyle{same}

% For \email{ADDRESS}, links ADDRESS to the url mailto:ADDRESS
% (uncomment to typeset the e\-/mail address in typewriter font;
%  otherwise, will be typeset in the \urlstyle above)
%\DeclareUrlCommand\emaillink{\urlstyle{tt}}
\providecommand*\emaillink[1]{\nolinkurl{#1}}
\providecommand*\email[1]{\href{mailto:#1}{\emaillink{#1}}}

\providecommand\BibTeX{{B\kern-.05em{\sc i\kern-.025em b}\kern-.08em \TeX}}
\providecommand\Matlab{\textsc{Matlab}}

% Custom hyphenation rules for words that LaTeX has trouble with
\hyphenation{bio-mim-ic-ry bio-in-spi-ra-tion re-us-a-ble pro-vid-er Media-Wiki}

%%%%%%%%%%%%%%%%%%%%%%%% End Helper Commands %%%%%%%%%%%%%%%%%%%%%%%%%%%

%%%%%%%%%%%%%%%%%%%%%%%%% Begin CV Document %%%%%%%%%%%%%%%%%%%%%%%%%%%%

\begin{document}
	\makeheading{\LARGE{Aaron M. Mueller}}
	\\

	\section{Coordonnées}
	Center for Language \& Speech Processing
	\hfill{\textit{Courriel:} amueller@jhu.edu} \\
	Johns Hopkins University
	\hfill{\textit{Site web:} aaronmueller.github.io} \\
	3400 N. Charles St., Hackerman 319 \\
	Baltimore, MD 21218-2608 (É-U)
	
	%%
	%% In modern CV's, it seems like ``Objective'' is frowned upon. Instead,
	%% incorporate it into a well-constructed cover letter. The ``More
	%% information'' can go at the end of the CV, but it should not distract
	%% from the section giving references available to contact.
	%%
	%
	% \section{Objective}
	%
	% Placement in an academic position (i.e., faculty, postdoctoral, or
	% research scientist) that allows for advanced research in distributed
	% complex adaptive systems (i.e., modeling, analysis, design, and
	% verification) with a particular focus on the control of engineered
	% agents (e.g., for communications, control, software, electronics, and
	% sustainability) and the analysis of biological phenomena (e.g.,
	% self-organization, ecological rationality)
	% \begin{innerlist}
	% \item More information and auxiliary documents can be found at\\\url{http://www.tedpavlic.com/facjobsearch/}
	% \end{innerlist}
	
	\section{Intérêts de recherche}
	\begin{itemize}[topsep=0pt]
		\setlength{\itemindent}{-.2in}
		\itemsep-0.3em
		\item Psycholinguistique informatique
		\item Analyse des médias sociaux (sociolinguistique)
		\item Langues menacées et à faibles ressources
		\item Traduction automatique
	\end{itemize}

	\section{Formation}

	{\textbf{Johns Hopkins University}},
	Baltimore, MD, É-U
	\begin{outerlist}[topsep=0pt]
	\itemsep-0.3em
	\item[] Doctorant en informatique.
	\hfill Depuis août 2018
	\item[] Maîtrise en informatique.
	\hfill Attendu mai 2020
	\item[] \textit{Affiliation}: Center for Language and Speech Processing.
	\item[] \textit{Directeur de thèse}: David Yarowsky.\\
	\end{outerlist}

	{\textbf{Université du Kentucky}},
	Lexington, KY, É-U
	\begin{outerlist}[topsep=0pt]
		\itemsep-0.3em
		\item[] Baccalauréat en informatique (avec mention).
		\hfill Mai 2018
		\item[] Baccalauréat en linguistique (avec mention).
		\hfill Mai 2018
		\item[] GPA: 4.0/4.0. \textit{Summa cum laude}.
		\item[] \emph{Thèse}: Neural Machine Translation for Canadian French.
		\item[] \emph{Directeurs de thèse}: Ramakanth Kavuluru, Mark Richard Lauersdorf.
	\end{outerlist}

	\halfblankline
		
	\section{Expérience\\en recherche}

	{\textbf{BBN Technologies}},
	Cambridge, MA, É-U\\
		\textit{Stagiaire en recherche linguistique et multimédia}
		\hfill Mai - août 2019\\
		Superviseurs: Roger Bock, Ilana Heintz
		\begin{innerlist}
			\item Recherche en traduction automatique neuronale, alignement des mots et annotation sémantique.
			\item Financé par le programme AIDA (Active Interpretation of Disparate Alternatives) de la DARPA.
		\end{innerlist}

	\halfblankline

	{\textbf{Université du Kentucky}},
	Lexington, KY, É-U\\
	\textbf{Institutional Research \& Advanced Analytics}\\
		\textit{Assistant de recherche en apprentissage automatique}
		\hfill Mai - août 2018\\
		Superviseurs: Nathan Jacobs, Craig Rudick
		\begin{innerlist}
			\item Dessiner un système qui prédit les notes futures d'un étudiant donné sa performance précédente.
			\item Mettre en œuvre les LSTMs profonds et employer les techniques TALN pour apprendre les profils étudiants et cours.
			\item Dessiner une fonction objectif ordinale pour améliorer la performance du système.
		\end{innerlist}

	\halfblankline

	{\textbf{University du Massachusetts à Amherst}},
	Amherst, MA, É-U\\
	\textbf{Statistical Social Language Analysis Lab}\\
		\textit{Assistant de recherche en TALN}%
		\hfill Mai - août 2017\\
		Superviseur: Brendan O'Connor
		\begin{innerlist}
			\item Collaborer avec des doctorants et professeurs pour améliorer un système d'extraction des entités et évènements.
			\item Donné des articles de nouvelles, utiliser la supervision distante pour soutirer les noms des citoyens américains qui a été tué par la police.
			\item Intégrer les systèmes d'extraction des relations des entités dans le modéle préexistant au niveau des phrases complètes.
			\item Intégrer l'information au niveau du document (p. ex., les coréférences de noms) comme caractéristiques pour améliorer notre système.
		\end{innerlist}
	
	\halfblankline
	
	{\textbf{Université du Kentucky}},
	Lexington, KY, É-U\\
	\textbf{Département de linguistique}\\
		\textit{Projet de jury}%
		\hfill Août 2016 - mai 2017\\
		Jurés: Hilaria Cruz, Raphael Finkel, Phil Harling
		\begin{innerlist}
			\item Projet de préservation et revitalisation pour une langue autochtone à faible ressources---Chatino---parlée par 40 000 personnes.
			\item Entraîner un système de reconnaissance automatique de la parole avec Sphinx.
			\item Promouvoir et créer des ressources pour l'apprentissage des langues chatino, dont les données source-ouvert de parole, un corpus de parole et un site web.
		\end{innerlist}
	
	\halfblankline
	
	{\textbf{Johns Hopkins University}},
	Baltimore, MD, É-U\\
	\textbf{Center for Language \& Speech Processing}\\
		\textit{Assistant de recherche en TALN}%
		\hfill Mai - août 2016\\
		Superviseur: David Yarowsky
		\begin{innerlist}
			\item Collaborer avec des doctorants et professeurs pour créer un système de traduction automatique statistique pour les langues à faibles ressources.
			\item Mettre en œuvre un modèle de traduction anglais-ouïghour basé sur les lemmes.
			\item Construire un générateur morphologique pour le tatar de Crimée.
			\item Créer des analyseurs syntaxiques pour soutirer les tables de traduction des éditions étrangères de Wiktionnaire. 
			\item Financé par le programme LORELEI (Low-Resource Languages for Emergent Incidents) de la DARPA.
		\end{innerlist}

	\section{Bourses et prix}

	\textbf{Boursier de recherche de la National Science Foundation}
	\hfill 2018 - 2023
	
	\halfblankline
	
	\textbf{Boursier Gaines}
	\hfill 2016 - 2018\\
		Bourse de recherche de deux ans décernée aux étudiants de premier cycle pour leur rendement scolaire, capacité de mener des recherches, intérêt pour les affaires publiques et désir de faire avancer les connaissances de la condition humaine à travers les sciences humaines. Elle nécessite l'achèvement d'un projet de jury, un thèse et un cours séminaire en sciences humaines. (\$5 000 USD)
	
	\halfblankline

	\textbf{Boursier Patterson}
	\hfill 2014 - 2018\\
		Décernée aux étudiants à l'Université du Kentucky qui ont gagné le Bourse national du mérite.\\(\$80 000+ USD)

	\halfblankline

	\textbf{Boursier Raymond F. Betts}
	\hfill 2017\\
		Décernée à un étudiant par an pour aider ses travaux de thèse. Utilisé les fonds pour étudier le français québécois à Montréal et à Québec pendant l'hiver de 2017--2018. (\$2 500 USD)

	\halfblankline

	\textbf{Bourse Goldwater (Mention honorable)}
	\hfill 2017
	
	\halfblankline
	
	\textbf{Phi Beta Kappa}
	\hfill 2017
	
	\halfblankline

	\textbf{Prix de recherche en linguistique}
	\hfill 2016\\
		Attribué à un étudiant à l'Université du Kentucky pour faciliter un projet de recherche d'une année en linguistique. Utilisé les fonds pour un projet de sociolinguistique quantitative examinant les glissements sémantiques des mots sexospécifiques dans les médias américains sur 200 ans. (\$500 USD)

	\halfblankline
	
	\textbf{Bourse nationale du mérite}
	\hfill 2014
	
	\section{Présentations représent-\\atives}

	Mueller, Aaron. Québec French and Language Technology. \emph{Cérémonie de Boursiers Betts \& Rowland 2018}, Lexington, KY, É-U, 4 avril 2018.

	\halfblankline

	Mueller, Aaron; Keith, Katherine; Handler, Abe; Blodgett, Su Lin; et O'Connor, Brendan. The Identification of Civilians Killed by Police with Supervised Entity-Event Extraction. \emph{2017 UMass Amherst Research Experience for Undergraduates (REU) Showcase}, Amherst, MA, É-U, 9 août 2017.

	\halfblankline

	Mueller, Aaron; Finkel, Raphael; et Cruz, Hilaria. Documenting and Promoting the Chatino Language and Orthography. \emph{Présentation de jury dans la satisfaction des besoins de la bourse Gaines}, Lexington, KY, É-U, 21 fevrier 2017.
	
	\halfblankline

	Mueller, Aaron. A Lemma-Based Approach for English-Uyghur Statistical Machine Translation. \emph{9th Annual Conference of the Illinois Language and Linguistics Society (ILLS9)}, Urbana, IL, É-U, 31 mars--1 avril 2017.
	
	\halfblankline

	Mueller, Aaron et Lauersdorf, Mark R. Lexical and Semantic Shifts in the Linguistic Construction of Social Gender: A Corpus-Based Analysis of Written U.S. English. Poster.
	\emph{9th Annual Toronto Undergraduate Linguistics Conference (TULCON9)}, Toronto, ON, Canada, 4--6 mars 2016.

	\section{Expérience supplémen-\\taire}

	{\textbf{Chellgren Center for Undergraduate Excellence}},
	Lexington, KY, É-U\\
		\textit{Équipe de sensibilisation des étudiants}
		\hfill Août 2017 - mai 2018
		\begin{innerlist}
			\item Promouvoir le Bureau des recherches de $1^{er}$ cycle et la Bourse Chellgren.
			\item Organiser les tables rondes, ateliers et lectures sur l'écriture des résumés, le dessin des posters de recherches et les présentations académiques.
			\item Diriger le comité d'examen pour la Showcase of Undergraduate Scholars, une conférence universitaire annuelle pour les étudians de l'Université du Kentucky de présenter leurs recherches
			aux autres étudiants et professeurs.
		\end{innerlist}

	\halfblankline

	{\textbf{UK Special Collections Research Center}},
	Lexington, KY, É-U\\
		\textit{Stagiaire en archivage}%
		\hfill Septembre 2015 - mai 2016
		\begin{innerlist}
			\item Créer un corpus diachronique de documentation architecturale de Lexington, dont les ordonnaces de zonage et les procès-verbaux du comité organisateur de Lexington.
			\item Rendre accessible une collection d'archives sur l'histoire architecturale de Lexington.
			\item Coopérer sur un projet multimédia, dont les logiciels SIG, données de municipalité, photos d'archives, et outils d'humanités numériques.
		\end{innerlist}
	
	\section{Volontariat}
	
	\textbf{Amnesty International}
	\hfill Depuis 2012
	\begin{innerlist}
		\item Organisation internationale dévouée à la protection des droits de l'homme.
		\item Membre d'Amnesty International à Johns Hopkins et à Washington, D.C.
		\item \textit{Fondateur} et ancien \textit{coprésident} d'Amnesty International à l'Université du Kentucky.
		\item Organiser les collectes de fonds et manifestations d'information : les concerts avec des groupes locaux, tables rondes sur les questions contemporaines, débats de faculté, pétitions, protestations et plus.
		\item Participant à la Conference nationale d'Amnesty International sur les droits de l'homme.
	\end{innerlist}
	
	\halfblankline
	
	\textbf{Oxfam International}
	\hfill 2014 - 2018
	\begin{innerlist}
		\item Organisation internationale dévouée à l'élimination de la pauvreté, les secours aux sinistrés, and l'action politique pour les victimes de l'injustice dans le monde entier.
		\item \textit{Fondateur} and ancien \textit{co-coordonnateur} d'Oxfam à l'Université du Kentucky.
		\item Planifier les collectes de fonds, manifestations de sensibilisation et campagnes de signature des pétitions pour la campagne GROW d'Oxfam.
	\end{innerlist}

	\halfblankline
	
	\textbf{Société pour la promotion de la recherche de premier cycle}
	\hfill 2014 - 2017
	\begin{innerlist}
		\item Organisation basée à l'Université du Kentucky fondée pour aider les étudiants de $1^{er}$ cycle à trouver les opportunités de recherche  et à améliorer leurs compétences en recherche.
		\item \textit{Coprésident} et ancien \textit{Directeur des évènements}.
		\item Planifier and exécuter la Showcase of Undergraduate Scholars.
		\item Planifier les ateliers de recherche pour aider les étudiants à améliorer leurs résumés, présentations et documents académiques.
		\item Organiser les tables rondes de faculté, évènements promotionnels et réunions d'information sur s'impliquant dans la recherche, les bourse d'études et les opportunités de recherche d'été.
		\item Fusionné avec l'Équipe de sensibilisation des étudiants en mai 2017 (rejoint dès la fusion).
	\end{innerlist}

	\section{Compétences informatiques}
	
	Programmation:
	%
	\begin{innerlist}
		\item Langages: Python, C, C$+$$+$, Java, HTML, CSS, Javascript, \Matlab, R, script shell UNIX.
		\item Toolkits d'apprentissage automatique: TensorFlow, Sockeye, Keras, PyTorch, OpenNMT.
		\item Contrôle de version: DVCS (Git, Bitbucket).
		\item Environnements de développement: Vim, Emacs, Eclipse.
	\end{innerlist}
	
	\halfblankline
	
	Outils linguistiques:
	\begin{innerlist}
		\item[] Praat, AntConc, QGIS, Audacity.
	\end{innerlist}

	\halfblankline
	
	Logiciel bureautique de montage et productivité:
	%
	\begin{innerlist}
		\item \TeX{} (\LaTeX{}, \BibTeX{}).
		\item GIMP, Inkscape.
		\item Microsoft Office, G Suite, OpenOffice.org, LibreOffice
	\end{innerlist}

	\section{Langues}
	Français canadien (B2\textasciitilde C1), anglais (langue maternelle), inuktitut (rudimentaire). Expérience en chatino et ouïghour par la recherche.

	\section{Intérêts et loisirs}

	\begin{itemize}[topsep=0pt]
		\setlength{\itemindent}{-.2in}
		\itemsep-0.3em
		\item[--] Musique: guitare (classique, électrique), basse, banjo,
		synthétiseur, MaxMSP, Ableton Live.
		\item[--] Athlétisme: randonnée d'hiver, danse disco et swing.
		\item[--] Loisirs: apprentissage du langage naturel, lecture (philosophie, politique), exploration urbaine.
		\item[--] Compétences culinaires: thaïlandais, français, allemand, sud des États-Unis.
	\end{itemize}

	%\section{References Available to Contact}
	%
	%\textbf{Dr. Mark Richard Lauersdorf}
	%(e\-/mail:~lauersdorf@uky.edu; phone:~+1-859-257-7101)
	%
	%\begin{innerlist}
	%	\item Associate Professor of Linguistics,
	%	University of Kentucky
		
		%\item[$\diamond$] Linguistics Program,
		%1215 Patterson Office Tower,
		%University of Kentucky,
		%Lexington, KY 40506-0027
		
	%	\item[$\star$] \emph{Dr.~Lauersdorf is a current thesis advisor, former sociolinguistics research advisor, and former professor of multiple classes of mine.}
	%\end{innerlist}
	
	%\halfblankline
	
	%\textbf{Dr. Jerzy Jaromczyk}
	%(e\-/mail:~jurek@cs.uky.edu; phone:~+1-859-257-1186)
	%
	%\begin{innerlist}
	%	\item Associate Professor of Computer Science,
	%	University of Kentucky
		
		%\item[$\diamond$] Davis Marksbury Building,
		%329 Rose Street,
		%Lexington, KY 40506-0633
		
	%	\item[$\star$] \emph{Dr.~Jaromczyk is a current computer science research advisor.}
	%\end{innerlist}
	
	%\halfblankline
	%
	%\textbf{Dr. Hilaria Cruz}
	%(e\-/mail:~hcr224@g.uky.edu)
	%
	%\begin{innerlist}
	%	\item Lyman T. Johnson Postdoctoral Fellow,
	%	University of Kentucky
		
		%\item[$\diamond$] Linguistics Program,
		%1215 Patterson Office Tower,
		%University of Kentucky,
		%Lexington, KY 40506-0027
		
	%	\item[$\star$] \emph{Dr.~Cruz juried a language revitalization project of mine. She is a native speaker of Chatino, a low-resource indigenous language.}
	%\end{innerlist}
	
	%\halfblankline
	
	% The ``More Info'' section may not be necessary; make sure it's short
	% so it doesn't prevent people from seeing references available to
	% contact.
	%\section{More Information}
	
	%More information can be found at\\	\url{https://www.linkedin.com/in/aaron-mueller-8b9334104}.
	
\end{document}

%%%%%%%%%%%%%%%%%%%%%%%%%% End CV Document %%%%%%%%%%%%%%%%%%%%%%%%%%%%%

%----------------------------------------------------------------------%
% The following is copyright and licensing information for
% redistribution of this LaTeX source code; it also includes a liability
% statement. If this source code is not being redistributed to others,
% it may be omitted. It has no effect on the function of the above code.
%----------------------------------------------------------------------%
% Copyright (c) 2007, 2008, 2009, 2010, 2011 by Theodore P. Pavlic
%
% Unless otherwise expressly stated, this work is licensed under the
% Creative Commons Attribution-Noncommercial 3.0 United States License. To
% view a copy of this license, visit
% http://creativecommons.org/licenses/by-nc/3.0/us/ or send a letter to
% Creative Commons, 171 Second Street, Suite 300, San Francisco,
% California, 94105, USA.
%
% THE SOFTWARE IS PROVIDED "AS IS", WITHOUT WARRANTY OF ANY KIND, EXPRESS
% OR IMPLIED, INCLUDING BUT NOT LIMITED TO THE WARRANTIES OF
% MERCHANTABILITY, FITNESS FOR A PARTICULAR PURPOSE AND NONINFRINGEMENT.
% IN NO EVENT SHALL THE AUTHORS OR COPYRIGHT HOLDERS BE LIABLE FOR ANY
% CLAIM, DAMAGES OR OTHER LIABILITY, WHETHER IN AN ACTION OF CONTRACT,
% TORT OR OTHERWISE, ARISING FROM, OUT OF OR IN CONNECTION WITH THE
% SOFTWARE OR THE USE OR OTHER DEALINGS IN THE SOFTWARE.